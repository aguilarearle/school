\documentclass[11pt, letterpaper, onecolumn]{article}

% Imports
\usepackage[english]{babel}
\usepackage{fancyhdr}
\usepackage{extramarks}
\usepackage{amsmath, amsthm, amsfonts,mathtools, framed}
\usepackage[shortcuts]{extdash} % Use \-/ for hyphenated words
\usepackage{environ}
\usepackage{fancyvrb}
\usepackage[top=1.00in, bottom=1.00in, left=0.75in, right=0.75in]{geometry}
\usepackage{algorithm}
\usepackage{algpseudocode}
\usepackage{accents}
% Pseudocode enabler
\usepackage{listings}
\usepackage{color}

\usepackage{pdfpages}

\definecolor{dkgreen}{rgb}{0,0.6,0}
\definecolor{gray}{rgb}{0.5,0.5,0.5}
\definecolor{mauve}{rgb}{0.58,0,0.82}

\lstset{frame=tb,
  language=Java,
  aboveskip=3mm,
  belowskip=3mm,
  showstringspaces=false,
  columns=flexible,
  basicstyle={\small\ttfamily},
  numbers=none,
  numberstyle=\tiny\color{gray},
  keywordstyle=\color{blue},
  commentstyle=\color{dkgreen},
  stringstyle=\color{mauve},
  breaklines=true,
  breakatwhitespace=true,
  tabsize=3
}

% tikz
\usepackage{pgf}
\usepackage{tikz}
\usetikzlibrary{arrows,automata}

% Page
\headsep=0.3in
\linespread{1.15}
\setlength\parindent{0pt}

% Horizontal Rules
\newcommand{\hwRuleWidth}{0.4pt}
\renewcommand\headrulewidth{\hwRuleWidth}
\renewcommand\footrulewidth{\hwRuleWidth}

% Header and Footer
\pagestyle{fancy}
\lhead{\hwAuthor\ (\hwSection)}
\chead{\hwClass\ (\hwInstructor): \hwTitle}
\rhead{\firstxmark}
\lfoot{\lastxmark}
\cfoot{\thepage}
\newcommand{\enterProblemHeader}[2]{
  \nobreak\extramarks{}{Problem \arabic{#1}.\arabic{#2} continued on next page\ldots}\nobreak{}
  \nobreak\extramarks{Problem \arabic{#1}.\arabic{#2} (continued)}{Problem \arabic{#1}.\arabic{#2} continued on next page\ldots}\nobreak{}
}
\newcommand{\exitProblemHeader}[2]{
  \nobreak\extramarks{Problem \arabic{#1}.\arabic{#2} (continued)}{Problem \arabic{#1}.\arabic{#2} continued on next page\ldots}\nobreak{}
  \nobreak\extramarks{Problem \arabic{#1}.\arabic{#2}}{}\nobreak{}
}

% Counters
\setcounter{secnumdepth}{0}
\newcounter{partCounter}
\setcounter{partCounter}{1}
\newcounter{hwProblemCounter}
\newcounter{hwSubProblemCounter}
\numberwithin{equation}{hwProblemCounter}
\numberwithin{figure}{hwProblemCounter}

% Environments
\newenvironment{problem}[3] {
  \setcounter{hwProblemCounter}{#2}
  \setcounter{hwSubProblemCounter}{#3}
  \setcounter{partCounter}{1}
  \enterProblemHeader{hwProblemCounter}{hwSubProblemCounter}

  \large{\textbf{Problem #2.#3: #1}}\\
}{
  \exitProblemHeader{hwProblemCounter}{hwSubProblemCounter}
}

\newcommand{\question}[2]{\textbf{(#1)\ }\ #2\\}

\newenvironment{Proof}[1][Proof]
  {\proof[#1]\leftskip=1cm\rightskip=1cm}
  {\endproof}
%-------------------------------------------------------------------------------
% Assignmnet Variables
%-------------------------------------------------------------------------------
\newcommand{\hwTitle}{Homework 2}
\newcommand{\hwDueDate}{April XX, 2017}
\newcommand{\hwClass}{Statistics 100C}
\newcommand{\hwInstructor}{Nicolas Christou}
\newcommand{\hwAuthor}{Earle Aguilar}
\newcommand{\hwSection}{804501476}
% First Problem
\setcounter{hwProblemCounter}{1}
\setcounter{hwSubProblemCounter}{0}

%-------------------------------------------------------------------------------
\begin{document}
%-------------------------------------------------------------------------------
% TITLE
%-------------------------------------------------------------------------------
{\centering
  \Large{\textbf{\hwTitle}}\\
  \vspace{0.1in}\normalsize{\hwDueDate}\\
  \vspace{0.1in}\textbf{\hwAuthor} (\hwSection)\\
  \vspace{0.1in}\rule{\textwidth}{\hwRuleWidth}}\\


\newcommand{\msum}{\displaystyle \sum}
\newcommand{\betah}{\hat{\beta}}
\newcommand{\pmatb}{\begin{pmatrix}}
\newcommand{\pmate}{\end{pmatrix}}
%-------------------------------------------------------------------------------
% Problem 1.0
%------------------------------------------------------------------------------
\begin{problem}{}{1}{0}
  \begin{align}
    Q &=\pmatb \betah \\ \hat{Y} \pmate = \pmatb (X^{'}X)^{-1}X^{'}Y \\ (I-H)Y \pmate \\
    Q &= LY
  \end{align}
  
  \begin{align}
    Var(Q) &= Var(LY) = L Var(Y) L^{'} \\
           &= \sigma^2 L L^{'} \\
           &=\sigma^2 \pmatb (X^{'}X)^{-1}X^{'} \\ (I-H)\pmate 
    \pmatb (X^{'}X)^{-1}X^{'} & (I-H) \pmate \\
           &= \sigma^2 \pmatb (X^{'}X)^{-1} & (X^{'}X)^{-1}X^{'}(I-H) \\ (I-H)X(X^{'}X)^{-1} & (I-H) \pmate  \\
    Var(Q) &= \sigma^2 \pmatb (X^{'}X)^{-1} & 0 \\ 0 & (I-H)\pmate
  \end{align}
Therefore $\hat{Y}$ and $e$ are independent.
\end{problem}
%-------------------------------------------------------------------------------
% Problem 2.0
%-------------------------------------------------------------------------------
\begin{problem}{}{2}{0}
\end{problem}

%-------------------------------------------------------------------------------
% Problem 3.0
%-------------------------------------------------------------------------------
\begin{problem}{}{3}{0}
  We have that $S^2 = \frac{\msum_{i=1}^n (y_i-\bar{y})^2}{n-1}$ and 
  $S_e^2 = \frac{e^{'}e}{n-k-1}=
  \frac{\msum_{i=1}^n(y_i-\bar{y})^2-\msum_{i=1}^n(\hat{y_i}-\bar{y})^2}{n-k-1}$
  \begin{align}
    R_a^2 &= \frac{S^2-s_e^2}{S^2} \\
          &= \frac{n-1}{\msum_{i=1}^n(y_i-\bar{y})^2} 
            \bigg[\frac{\msum_{i=1}^n(y_i-\bar{y})^2}{n-1} - 
            \frac{\msum_{i=1}^n(y_i-\bar{y})^2}{n-k-1} + 
            \frac{\msum_{i=1}^n(\hat{y_i}-\bar{y})^2}{n-k-1} \bigg] \\
          &= \frac{n-1}{\msum_{i=1}^n(y_i-\bar{y})^2} 
            \bigg[\frac{(n-k-1)\msum_{i=1}^n(y_i-\bar{y})^2 -
            (n-1)\msum_{i=1}^n(y_i-\bar{y})^2 + 
            (n-1)\msum_{i=1}^n(\hat{y_i}-\bar{y})^2}{(n-1)(n-k-1)}\bigg] \\
          &= \bigg[\frac{(n-k-1) - (n-1) + 
            (n-1)\frac{\msum_{i=1}^n(\hat{y_i}-\bar{y})^2}{\msum_{i=1}^n(y_i-\bar{y})^2}}{n-k-1}\bigg] \\
          &= 1 - \frac{(n-1) - (n-1)R^2}{n-k-1} \\
    \Aboxed{R_a^2 &= 1 - \frac{n-1}{n-k-1}(1-R^2)}
  \end{align}
\end{problem}
%-------------------------------------------------------------------------------
% Problem 4.0
%-------------------------------------------------------------------------------
\begin{problem}{}{4}{0}
\end{problem}
% -------------------------------------------------------------------------------
% Problem 5.0
%-------------------------------------------------------------------------------
\begin{problem}{}{5}{0}
  \begin{align}
    (Y-Xc)^{'}(Y-Xc) - (Y-X\betah)^{'}(Y-X\betah) \\
    Y^{'}Y-Y^{'}Xc-c^{'}X^{'}Y+c^{'}X^{'}Xc-Y^{'}Y - Y^{'}Y + Y^{'}X\betah +
    \betah^{'}X^{'}Y - \betah^{'}X^{'}X\betah \\
  \end{align}
We use the normal equations to substitute for $Y^{'}X$ and $X^{'}Y$.
  \begin{align}
    -\betah^{'}X^{'}Xc-c^{'}X^{'}X\betah + c^{'}X^{'}Xc + \betah X^{'}X \betah +
    \betah^{'}X^{'}X\betah - \betah X^{'}X \betah \\\
    c^{'}X^{'}Xc -\betah^{'}X^{'}Xc-c^{'}X^{'}X\betah + \betah^{'}X^{'}X\betah \\
    (c^{'}X^{'}X-\betah^{'}X^{'}X) (c-\betah) \\
    (c^{'}- \betah^{'})X^{'}X (c-\betah) \\
    \Aboxed{(c- \betah)^{'}X^{'}X (c-\betah)}
  \end{align}
\end{problem}
%-------------------------------------------------------------------------------
% Problem 6.0
%-------------------------------------------------------------------------------
\begin{problem}{}{6}{0}
  \section{a}
  \begin{align}
    \betah &= \pmatb \betah_0 \\ \betah_1 \\ \betah_2 \pmate = (X^{'}X)^{-1}X^{'}Y \\
           &= \pmatb 2 & 0 & 0 \\ 0 & 3 & -1 \\ 0 & -1 & 1 \pmate \pmatb 4 \\ -2 \\ 5 \pmate \\
    \pmatb \betah_0 \\ \betah_1 \\ \betah_2 \pmate &= \pmatb 8 \\ -11 \\ 7\pmate
  \end{align}
\section{b}
\begin{align}
  X = 
  \pmatb
  1 & 3  & 1   & -1  & 1   \\
  1 & 4  & 1   & 1   & -1  \\
  1 & 5  & -1  & 1   & 1   \\
  1 & 6  & 0.5 & 0.2 & 0.3 \\
  1 & 8  & 0.8 & 0.1 & 0.1 \\
  1 & 9  & 0.3 & 0.5 & 0.2 \\
  1 & 10 & 0.2 & 0.3 & 0.5 \\
  1 & 13 & 0.1 & 0.6 & 0.3 \\
  \pmate
\end{align}
$x_2 + x_3 + x_4 = 1 \implies$ matrix is not full rank and we cannot invert $X^{'}X$.
\end{problem}
%-------------------------------------------------------------------------------
% Problem 7.0
%-------------------------------------------------------------------------------
\begin{problem}{}{7}{0}
\end{problem}
\end{document}


 
% \begin{align}
% \end{align}
% \frac{}{}
