\documentclass[11pt, letterpaper, onecolumn]{article}

% Imports
\usepackage[english]{babel}
\usepackage{fancyhdr}
\usepackage{extramarks}
\usepackage{amsmath, amsthm, amsfonts,mathtools, framed, wasysym}
\usepackage[shortcuts]{extdash} % Use \-/ for hyphenated words
\usepackage{environ}
\usepackage{fancyvrb}
\usepackage[top=1.00in, bottom=1.00in, left=0.75in, right=0.75in]{geometry}
\usepackage{algorithm}
\usepackage{algpseudocode}
\usepackage{accents}
% Pseudocode enabler
\usepackage{listings}
\usepackage{color}

\definecolor{dkgreen}{rgb}{0,0.6,0}
\definecolor{gray}{rgb}{0.5,0.5,0.5}
\definecolor{mauve}{rgb}{0.58,0,0.82}

\lstset{frame=tb,
  language=Java,
  aboveskip=3mm,
  belowskip=3mm,
  showstringspaces=false,
  columns=flexible,
  basicstyle={\small\ttfamily},
  numbers=none,
  numberstyle=\tiny\color{gray},
  keywordstyle=\color{blue},
  commentstyle=\color{dkgreen},
  stringstyle=\color{mauve},
  breaklines=true,
  breakatwhitespace=true,
  tabsize=3
}

% tikz
\usepackage{pgf}
\usepackage{tikz}
\usetikzlibrary{arrows,automata}

% Page
\headsep=0.3in
\linespread{1.15}
\setlength\parindent{0pt}

% Horizontal Rules
\newcommand{\hwRuleWidth}{0.4pt}
\renewcommand\headrulewidth{\hwRuleWidth}
\renewcommand\footrulewidth{\hwRuleWidth}

% Header and Footer
\pagestyle{fancy}
\lhead{\hwAuthor\ (\hwSection)}
\chead{\hwClass\ (\hwInstructor): \hwTitle}
\rhead{\firstxmark}
\lfoot{\lastxmark}
\cfoot{\thepage}
\newcommand{\enterProblemHeader}[2]{
  \nobreak\extramarks{}{Problem \arabic{#1}.\arabic{#2} continued on next page\ldots}\nobreak{}
  \nobreak\extramarks{Problem \arabic{#1}.\arabic{#2} (continued)}{Problem \arabic{#1}.\arabic{#2} continued on next page\ldots}\nobreak{}
}
\newcommand{\exitProblemHeader}[2]{
  \nobreak\extramarks{Problem \arabic{#1}.\arabic{#2} (continued)}{Problem \arabic{#1}.\arabic{#2} continued on next page\ldots}\nobreak{}
  \nobreak\extramarks{Problem \arabic{#1}.\arabic{#2}}{}\nobreak{}
}

% Counters
\setcounter{secnumdepth}{0}
\newcounter{partCounter}
\setcounter{partCounter}{1}
\newcounter{hwProblemCounter}
\newcounter{hwSubProblemCounter}
\numberwithin{equation}{hwProblemCounter}
\numberwithin{figure}{hwProblemCounter}

% Environments
\newenvironment{problem}[3] {
  \setcounter{hwProblemCounter}{#2}
  \setcounter{hwSubProblemCounter}{#3}
  \setcounter{partCounter}{1}
  \enterProblemHeader{hwProblemCounter}{hwSubProblemCounter}

  \large{\textbf{Problem #2.#3: #1}}\\
}{
  \exitProblemHeader{hwProblemCounter}{hwSubProblemCounter}
}

\newcommand{\question}[2]{\textbf{(#1)\ }\ #2\\}

\newenvironment{Proof}[1][Proof]
  {\proof[#1]\leftskip=1cm\rightskip=1cm}
  {\endproof}
%-------------------------------------------------------------------------------
% Assignmnet Variables
%-------------------------------------------------------------------------------
\newcommand{\hwTitle}{Homework 4}
\newcommand{\hwDueDate}{Feb 19, 2017}
\newcommand{\hwClass}{CS 181}
\newcommand{\hwInstructor}{Amit Sahai}
\newcommand{\hwAuthor}{}
\newcommand{\hwSection}{804501476}

% First Problem

\setcounter{hwProblemCounter}{1}
\setcounter{hwSubProblemCounter}{0}

%-------------------------------------------------------------------------------
\begin{document}
%-------------------------------------------------------------------------------
% TITLE
%-------------------------------------------------------------------------------
{\centering
  \Large{\textbf{\hwTitle}}\\
  \vspace{0.1in}\normalsize{\hwDueDate}\\
  \vspace{0.1in}\textbf{\hwAuthor} (\hwSection)\\
  \vspace{0.1in}\rule{\textwidth}{\hwRuleWidth}}\\


%-------------------------------------------------------------------------------
% Problem 1.0
%------------------------------------------------------------------------------
\begin{problem}{}{1}{0}
  \section{a)}
  $L_1 = \{0^a 1^b \ | \ a \ \text{divides} \ b \}$ is equivalent to
  $L_1 = \{ 0^a 1^{am} \ | \ m \ge 0 \}$.
 
  \begin{proof} $ $ \\
    Towards contradiction: \\
    - Assume $L_1$ is CF. \\
    - By pumping lemma $\exists$ pumping length p. \\
    - Let $s=0^p1^{pm}$ s.t. $|s| \ge p$. \\
    - By pumping lemma $0^p1^{pm} = xyzuv $ \\
    1) $|yzu| \le p$ \\
    2) $|yu| > 0$ \\
    \textbf{Cases}: \\
    1) $yzu = 0^k$, $1 \le k \le p$. Then $xy^2zu^2v = 0^{p+\alpha} 1^{pm}$ \\
    \begin{align}
      \frac{a}{b} &= m \\
      \frac{pm}{p+\alpha} &= m \\
      pm &= pm + m\alpha \\
      m\alpha = 0
    \end{align} 
    $\alpha$ cannot be zero. \\
    2) $yzu = 1^{\alpha}$ then $xzv = 0^p 1^{pm-\alpha}$
    \begin{align}
      \frac{a}{b} &= m \\
      \frac{pm-\alpha}{p} &= m \\
      pm-\alpha = pm \\
      \alpha = 0
    \end{align}
    $\alpha$ cannot be zero. \\
    3) $yzu = 0^{\alpha} 1^{\beta}$, $1 \le \alpha + \beta \le p$ then $xy^2zv^2u = 0^{p+\alpha}1^{pm+\beta}$
    \begin{align}
      \frac{a}{b} &= m \\
      \frac{pm+\beta}{p+\alpha} &= m \\
      pm+\beta = pm + m\alpha  \\
      \beta = m\alpha
    \end{align}
  \end{proof}
  \section{b)}
\end{problem}
%-------------------------------------------------------------------------------
% Problem 2.0
%-------------------------------------------------------------------------------
\begin{problem}{}{2}{0}
  \section{a)}
  We have two stacks $S_1$ and $S_2$. First the elements of a string must be pushed
  onto $S_1$. Once all elements are in $S_1$ push all but the bottom most
  element onto $S_2$. Now the stacks will represent a window into two elements of
  of the input string, the top of $S_1$ will be the element on the left and
  the top of $S_2$ will be the element on the right. To simulate head movement
  of Turing machien T we define left and right movement by pushing and popping from stacks of
  multi-stack machine M. \\ \\
  \textbf{Move head to the right} \\
  $Push_{S_1}(Pop_{S_2}) $ \\
  \textbf{Move head to the right and replace element with $\beta$} \\
  $Pop_{S_1}, Push_{S_1}(\beta),Push_{S_1}(Pop_{S_2}(\alpha)) $ \\
  \textbf{Move head to the right and write $\beta$} \\
  $Push_{S_1}(\beta),Push_{S_1}(Pop_{S_2}(\alpha)) $ \\
  \textbf{Move head to the left} \\
  $Push_{S_2}(Pop_{S_1}(\alpha)) $ \\
  \textbf{Move head to the left and replace element with $\beta$} \\
  $Pop_{S_1},Push_{S_2}(\beta) $ \\
  \textbf{Move head to the left and write $\beta$} \\
  $Push_{S_2}(Pop_{S_1}(\alpha)), Push_{s_2}(\beta) $  \\ \\
  The states of machine M are equal to the states of T. Transitioning between
  states is done using the $\delta$ function of machine T.  An accepting state
  of this machine would be conditioned on a specific configuration of the
  elements in the stacks and we can move left to right to determine if the
  configuration is valid and then transitioning on to an accept state that is in
  T.  
  \section{b)}
  We simulate three stacks using a turing machine T. In class we showed that
  a multi-tape turing machine can be simulated using a single tape. For
  simplicity we use multiple seperate tapes. We have an input tape to store
  the input string and $t_1$, $t_2$, and $t_3$ each tape for one stack. We are
  given machine M with three stacks $S_1$, $S_2$, and $S_3$. Let
  $x=x_1x_2\dots x_n$ be the input string. \\ \\
  \textbf{Starting States of Stacks and Tapes} \\
  Each stack has a starting state with an end of stack symbol on it. Machine T
  has the input $x$ on its input tape and each tape $t_i$ will have the
  corresponding stack symbol as its first symbol. \\
  \textbf{Stack Operations} \\
  Upon reading $x_i$ machine M can: \\
  1) $Push_{S_i}(x_i)$ - Push $x_i$ onto stack $S_i$ for $i \in \{1,2,3\}$ \\
  2) $Pop_{S_i}$ - Pop element off of $S_i$. \\
  3) $Push_{S_i}(Pop_{S_j}), \ i \ne j$  - Pop and element of of one stack and
  push to another. \\
  \textbf{Equivalent Tape Operations} \\
  Each operation corresponds to the operations defined above respectively.
  Upon reading $x_i$ machine T: \\
  1) Move head to end of tape $t_i$, when first \textvisiblespace \ is under head, and
  write $x_i$ onto it. \\
  2) Move head to end of tape $t_i$, then move the the left by one and write
  \textvisiblespace. \\
  3) Move to end of tape $t_j$ and the left by one. Then move to end of $t_i$
  and write symbol under head of $t_j$ onto $t_i$. \\
  
  The states of machine T are equal to the states of machine M. Transitions
  between states in T follow the $\delta$ function of machine M. T accepts if
  after processing string $x$ it is in the accept states of machine M.  \\ \\

  In part a) we showed that a Turing machine is no more powerful than a 2 stack
  PDA. In part b) we showed that a 3-stack machine is no more powerful than a
  TM machine. Therefore a 3-stack PDA is no more powerful than a 2-stack PDA.
\end{problem}
%-------------------------------------------------------------------------------
% Problem 3.0
%-------------------------------------------------------------------------------

\end{document}



 
% \begin{align}
% \end{align}
% \frac{}{}

