\documentclass[11pt, letterpaper, onecolumn]{article}

% Imports
\usepackage[english]{babel}
\usepackage{fancyhdr}
\usepackage{extramarks}
\usepackage{amsmath, amsthm, amsfonts,mathtools, framed, wasysym}
\usepackage[shortcuts]{extdash} % Use \-/ for hyphenated words
\usepackage{environ}
\usepackage{fancyvrb}
\usepackage[top=1.00in, bottom=1.00in, left=0.75in, right=0.75in]{geometry}
\usepackage{algorithm}
\usepackage{algpseudocode}
\usepackage{accents}
% Pseudocode enabler
\usepackage{listings}
\usepackage{color}

\definecolor{dkgreen}{rgb}{0,0.6,0}
\definecolor{gray}{rgb}{0.5,0.5,0.5}
\definecolor{mauve}{rgb}{0.58,0,0.82}

\lstset{frame=tb,
  language=Java,
  aboveskip=3mm,
  belowskip=3mm,
  showstringspaces=false,
  columns=flexible,
  basicstyle={\small\ttfamily},
  numbers=none,
  numberstyle=\tiny\color{gray},
  keywordstyle=\color{blue},
  commentstyle=\color{dkgreen},
  stringstyle=\color{mauve},
  breaklines=true,
  breakatwhitespace=true,
  tabsize=3
}

% tikz
\usepackage{pgf}
\usepackage{tikz}
\usetikzlibrary{arrows,automata}

% Page
\headsep=0.3in
\linespread{1.15}
\setlength\parindent{0pt}

% Horizontal Rules
\newcommand{\hwRuleWidth}{0.4pt}
\renewcommand\headrulewidth{\hwRuleWidth}
\renewcommand\footrulewidth{\hwRuleWidth}

% Header and Footer
\pagestyle{fancy}
\lhead{\hwAuthor\ (\hwSection)}
\chead{\hwClass\ (\hwInstructor): \hwTitle}
\rhead{\firstxmark}
\lfoot{\lastxmark}
\cfoot{\thepage}
\newcommand{\enterProblemHeader}[2]{
  \nobreak\extramarks{}{Problem \arabic{#1}.\arabic{#2} continued on next page\ldots}\nobreak{}
  \nobreak\extramarks{Problem \arabic{#1}.\arabic{#2} (continued)}{Problem \arabic{#1}.\arabic{#2} continued on next page\ldots}\nobreak{}
}
\newcommand{\exitProblemHeader}[2]{
  \nobreak\extramarks{Problem \arabic{#1}.\arabic{#2} (continued)}{Problem \arabic{#1}.\arabic{#2} continued on next page\ldots}\nobreak{}
  \nobreak\extramarks{Problem \arabic{#1}.\arabic{#2}}{}\nobreak{}
}

% Counters
\setcounter{secnumdepth}{0}
\newcounter{partCounter}
\setcounter{partCounter}{1}
\newcounter{hwProblemCounter}
\newcounter{hwSubProblemCounter}
\numberwithin{equation}{hwProblemCounter}
\numberwithin{figure}{hwProblemCounter}

% Environments
\newenvironment{problem}[3] {
  \setcounter{hwProblemCounter}{#2}
  \setcounter{hwSubProblemCounter}{#3}
  \setcounter{partCounter}{1}
  \enterProblemHeader{hwProblemCounter}{hwSubProblemCounter}

  \large{\textbf{Problem #2.#3: #1}}\\
}{
  \exitProblemHeader{hwProblemCounter}{hwSubProblemCounter}
}

\newcommand{\question}[2]{\textbf{(#1)\ }\ #2\\}

\newenvironment{Proof}[1][Proof]
  {\proof[#1]\leftskip=1cm\rightskip=1cm}
  {\endproof}
%-------------------------------------------------------------------------------
% Assignmnet Variables
%-------------------------------------------------------------------------------
\newcommand{\hwTitle}{Homework 4}
\newcommand{\hwDueDate}{Feb 19, 2017}
\newcommand{\hwClass}{CS 181}
\newcommand{\hwInstructor}{Amit Sahai}
\newcommand{\hwAuthor}{}
\newcommand{\hwSection}{804501476}

% First Problem

\setcounter{hwProblemCounter}{1}
\setcounter{hwSubProblemCounter}{0}

%-------------------------------------------------------------------------------
\begin{document}
%-------------------------------------------------------------------------------
% TITLE
%-------------------------------------------------------------------------------
{\centering
  \Large{\textbf{\hwTitle}}\\
  \vspace{0.1in}\normalsize{\hwDueDate}\\
  \vspace{0.1in}\textbf{\hwAuthor} (\hwSection)\\
  \vspace{0.1in}\rule{\textwidth}{\hwRuleWidth}}\\


%-------------------------------------------------------------------------------
% Problem 1.0
%------------------------------------------------------------------------------
\begin{problem}{}{1}{0}
  \section{a)}
  $L_1 = \{0^a 1^b \ | \ a \ \text{divides} \ b \}$ is equivalent to
  $L_1 = \{ 0^a 1^{am} \ | \ m \ge 0 \}$.
 
  \begin{proof} $ $ \\
    Towards contradiction: \\
    - Assume $L_1$ is CF. \\
    - By pumping lemma $\exists$ pumping length p. \\
    - Let $s=0^p1^{pm}$ s.t. $|s| \ge p$. \\
    - By pumping lemma $0^p1^{pm} = xyzuv $ \\
    1) $|yzu| \le p$ \\
    2) $|yu| > 0$ \\
    \textbf{Cases}: \\
    1) $yzu = 0^k$, $1 \le k \le p$. Then $xy^2zu^2v = 0^{p+\alpha} 1^{pm}$ \\
    \begin{align}
      \frac{a}{b} &= m \\
      \frac{pm}{p+\alpha} &= m \\
      pm &= pm + m\alpha \\
      m\alpha = 0
    \end{align} 
    $\alpha$ cannot be zero. \\
    2) $yzu = 1^{\alpha}$ then $xzv = 0^p 1^{pm-\alpha}$
    \begin{align}
      \frac{a}{b} &= m \\
      \frac{pm-\alpha}{p} &= m \\
      pm-\alpha = pm \\
      \alpha = 0
    \end{align}
    $\alpha$ cannot be zero. \\
    3) $yzu = 0^{\alpha} 1^{\beta}$, $1 \le \alpha + \beta \le p$ then $xy^2zv^2u = 0^{p+\alpha}1^{pm+\beta}$
    \begin{align}
      \frac{a}{b} &= m \\
      \frac{pm+\beta}{p+\alpha} &= m \\
      pm+\beta = pm + m\alpha  \\
      \beta = m\alpha
    \end{align}
  \end{proof}
  \section{b)}
\end{problem}
%-------------------------------------------------------------------------------
% Problem 2.0
%-------------------------------------------------------------------------------
\begin{problem}{}{2}{0} 
\end{problem}
%-------------------------------------------------------------------------------
% Problem 3.0
%-------------------------------------------------------------------------------
\begin{problem}{}{3}{0}
\end{problem}

\end{document}



 
% \begin{align}
% \end{align}
% \frac{}{}

