\documentclass[11pt, letterpaper, onecolumn]{article}

% Imports
\usepackage[english]{babel}
\usepackage{fancyhdr}
\usepackage{extramarks}
\usepackage{amsmath, amsthm, amsfonts,mathtools, framed}
\usepackage[shortcuts]{extdash} % Use \-/ for hyphenated words
\usepackage{environ}
\usepackage{fancyvrb}
\usepackage[top=1.00in, bottom=1.00in, left=0.75in, right=0.75in]{geometry}
\usepackage{algorithm}
\usepackage{algpseudocode}
\usepackage{accents}
% Pseudocode enabler
\usepackage{listings}
\usepackage{color}

\definecolor{dkgreen}{rgb}{0,0.6,0}
\definecolor{gray}{rgb}{0.5,0.5,0.5}
\definecolor{mauve}{rgb}{0.58,0,0.82}

\lstset{frame=tb,
  language=Java,
  aboveskip=3mm,
  belowskip=3mm,
  showstringspaces=false,
  columns=flexible,
  basicstyle={\small\ttfamily},
  numbers=none,
  numberstyle=\tiny\color{gray},
  keywordstyle=\color{blue},
  commentstyle=\color{dkgreen},
  stringstyle=\color{mauve},
  breaklines=true,
  breakatwhitespace=true,
  tabsize=3
}

% tikz
\usepackage{pgf}
\usepackage{tikz}
\usetikzlibrary{arrows,automata}

% Page
\headsep=0.3in
\linespread{1.15}
\setlength\parindent{0pt}

% Horizontal Rules
\newcommand{\hwRuleWidth}{0.4pt}
\renewcommand\headrulewidth{\hwRuleWidth}
\renewcommand\footrulewidth{\hwRuleWidth}

% Header and Footer
\pagestyle{fancy}
\lhead{\hwAuthor\ (\hwSection)}
\chead{\hwClass\ (\hwInstructor): \hwTitle}
\rhead{\firstxmark}
\lfoot{\lastxmark}
\cfoot{\thepage}
\newcommand{\enterProblemHeader}[2]{
  \nobreak\extramarks{}{Problem \arabic{#1}.\arabic{#2} continued on next page\ldots}\nobreak{}
  \nobreak\extramarks{Problem \arabic{#1}.\arabic{#2} (continued)}{Problem \arabic{#1}.\arabic{#2} continued on next page\ldots}\nobreak{}
}
\newcommand{\exitProblemHeader}[2]{
  \nobreak\extramarks{Problem \arabic{#1}.\arabic{#2} (continued)}{Problem \arabic{#1}.\arabic{#2} continued on next page\ldots}\nobreak{}
  \nobreak\extramarks{Problem \arabic{#1}.\arabic{#2}}{}\nobreak{}
}

% Counters
\setcounter{secnumdepth}{0}
\newcounter{partCounter}
\setcounter{partCounter}{1}
\newcounter{hwProblemCounter}
\newcounter{hwSubProblemCounter}
\numberwithin{equation}{hwProblemCounter}
\numberwithin{figure}{hwProblemCounter}

% Environments
\newenvironment{problem}[3] {
  \setcounter{hwProblemCounter}{#2}
  \setcounter{hwSubProblemCounter}{#3}
  \setcounter{partCounter}{1}
  \enterProblemHeader{hwProblemCounter}{hwSubProblemCounter}

  \large{\textbf{Problem #2.#3: #1}}\\
}{
  \exitProblemHeader{hwProblemCounter}{hwSubProblemCounter}
}

\newcommand{\question}[2]{\textbf{(#1)\ }\ #2\\}

\newenvironment{Proof}[1][Proof]
  {\proof[#1]\leftskip=1cm\rightskip=1cm}
  {\endproof}
%-------------------------------------------------------------------------------
% Assignmnet Variables
%-------------------------------------------------------------------------------
\newcommand{\hwTitle}{Homework 1}
\newcommand{\hwDueDate}{Jan 26, 2017}
\newcommand{\hwClass}{CS 181}
\newcommand{\hwInstructor}{Amit Sahai}
\newcommand{\hwAuthor}{}
\newcommand{\hwSection}{804501476}
% First Problem
\setcounter{hwProblemCounter}{1}
\setcounter{hwSubProblemCounter}{0}

%-------------------------------------------------------------------------------
\begin{document}
%-------------------------------------------------------------------------------
% TITLE
%-------------------------------------------------------------------------------
{\centering
  \Large{\textbf{\hwTitle}}\\
  \vspace{0.1in}\normalsize{\hwDueDate}\\
  \vspace{0.1in}\textbf{\hwAuthor} (\hwSection)\\
  \vspace{0.1in}\rule{\textwidth}{\hwRuleWidth}}\\


%-------------------------------------------------------------------------------
% Problem 1.0
%------------------------------------------------------------------------------
\begin{problem}{}{1}{0}
  \section{a} \textbf{$L_{Shuffle}(L_1, L_2)$ not regular} \\
  Let $L_1 = L_2 = \{ 0^n 1^n \}$ then
  \begin{align}
    L_{Shuffle}(L_1, L_2) &= \{ 0^n0^n1^n1^n\} \\
    L_{Shuffle}(L_1, L_2) & = \{0^{2n}1^{2n}\} = \{ 0^m 1^m | m = 2n\}
  \end{align}
  We know that $\{0^m1^m\}$ is not regular so $L_{Shuffle}(L_1, L_2)$ is not
  regular. \\ \\
  \textbf{$L_{Shuffle}(L_1, \bar{L_2})$ not regular} \\
  Let $L^{'}_2 = \{2^n3^n\} \subseteq \bar{L_2}$ then
  \begin{align}
    L_{Shuffle}(L_1, \bar{L_2}) &= \{ \{02\}^n \{13\}^n \} 
  \end{align}
  $\{\{02\}^n \{13\}^n\}$ is not regular and therefore
  $L_{Shuffle}(L_1, \bar{L_2})$ is also not regular.
  \section{b}
  Claim: For $x \in L_1$ and $y \in L_2$, s.t. $|x|=|y|=n$, $\exists$ machines
  $N_1$ and $N_2$ s.t $N_1(x): Accepts$ and $N_2(y): Accepts$ $\iff$ $\exists$
  M that recognizes $w \in L_{shuffle}(x,y)$, $M(w): Accepts$. 
  \begin{proof} $ $ \\
    Let M be a DFA recognizing $L_{shuffle}$:  
    \begin{itemize}
    \item $Q = Q_1 \cup Q_2$
    \item $q^{''}_0 = q_0 \in Q_1$
    \item $F^{''} = F^{'}$
    \item $\delta^{'}(s \in Q, w_i)  = \begin{cases} \delta_1(q_0, w_i), \ s=q_0 
      \\ \delta_2(s^{'}, w_i), \ s \in Q_1, \ s^{'} \in Q_2
      \\ \delta_1(s^{'}, w_i), \ s \in Q_2, \ s^{'} \in Q_1
    \end{cases}$
    \end{itemize}
    ($\Rightarrow$)   $L_1$ and $L_2$ be regular $\implies \exists$ DFAs $N_1$
    and $N_2$ that accept $x$ and $y$ respectively. Now $N_1$ recognizes x as
    follows. $\exists$ states $a_1, a_2, \dots, a_n$ s.t.
    $\delta_1(a_i, x_i) = \{a_j \in Q_1\}$ and $a_n \in F$. $N_2$ recognizes y
    since $\exists$ states $b_1, b_2, \dots, b_n$ s.t
    $\delta_2(b_i, y_i)=\{b_j \in Q_2\}$. Now the states $a_1b_1a_2b_2\dots a_nb_n$
    is concatenation of paths on $N_1$ and $N_2$ from $q_0 \in Q_1$ to
    $q_n \in Q_2$ for for $L_{shuffle}(x,y).$
    
    ($\Leftarrow$) 
    Now M recognizes $L_{Shuffle}$ as follows. Let
    w=$w_1w_2\dots w_m \in L_{shuffle}$, then $\exists$ states $s_1,...s_{2n}$ s.t
    $s_1 = a_1$ and $s_m = b_n$. Therefore a concatenation of their paths
    $a_1b_1a_2b_2\dots a_nb_n$ is an accpeting compuation path on M for w.
    
  \end{proof}

\end{problem}
%-------------------------------------------------------------------------------
% Problem 2.0
%-------------------------------------------------------------------------------
\begin{problem}{}{2}{0}
  \section{a}
  \begin{proof}
    Assume $L_2 = b^{*} \cup L_1$ is regular. Then by Pumping Lemma
    $\exists \  p \in \mathbb{N}$  s.t $w=b^p = xyz$, $|xy| \le p$ and
    $|y| \ge 1$. Now by Pumping Lemma we can
    pump so that $ab^ic \in L_2$
    since $b*$ is the set off all possible strings created by concatenating
    any number $i$ of $b$ it will produce an $ab^ic \in L_2$. 
  \end{proof}
  \section{b}
  \begin{proof}
    Let $M = (Q, \Sigma, \delta, q_0, F)$ be a DFA recognizing L and $n=k+p+m$
    be the number of states in M. Let $s=s_1s_2\dots s_n$ be a string in L s.t a partion 
    $\exists$ where $s=xyz$ as such: \\
    1) $x =s_1 \dots s_k$ \\ 
    2) $y = s_{1^{'}} \dots s_l$ s.t $|y| \ge p$ \\
    3) $z =s_{1^{''}} \dots s_m$ \\
    Let $abc$ be strings s.t $y=abc$. Observe $r_{1^{'}} \dots r_{l+1}$ the sequence of 
    states M enters while processing Y so $r_{i+1} = \delta(r_i, s_{i^{'}})$ for 
    $1^{'} \le i \le l$. The sequence has length $l+1$ which is at least $p+1$. By 
    the pigeonhole principle $\exists \ i,j$ s.t $r_i=r_j$, $1^{'} \le i < j \le l$. 
    \\ \\ \\ \\ \\ \\ \\
    Let $a=r_{1^{'}} \dots r_{i-1}$, $b=r_{i} \dots r_{j-1}$, $c=r_{j} \dots r_{l}$  
    Since $i \ne j \implies |b| > 0$ and since $x$ takes M $r_1$ to $r_{1^{'}}$, $a$
    takes M from $r_{'} \ to \ r_i$, $b$ takes M from $r_i \ to \ r_i$, $c$ takes M from
    $r_j \ to \ r_l$, and $z$ takes M from $r_{1^{''}} \ to \ r_m$ then $xab^{i}cz \in L$ 
  \end{proof}
  \section{c}
  \begin{proof} $ $ \\
    Assume for contradiction $L_2$ is regular
    \begin{itemize}
      \setlength\itemsep{0em}
      \item By Pumping Lemma $\exists \ p$ 
      \item Consider $\alpha^i\beta^n = xy$, where $y=\beta^n$ and $n > p$. 
      \item By pumping lemma $\beta^n = abc$, $|b| > 0$ 
      \item Let $|b| = q$ s.t $y=\beta^q\beta^{n-q}$ 
      \item Let $i = n+1 \implies \beta^{q(n+1)} \beta^{n-q} = \beta^{n(q+1)}$ 
      \item If $q=1$ then $2n$ is even or iff $q>1$ then $q+1|n(q+1)$ therefore
      \item $\alpha \beta^{n(q+1)} \notin L_2$ so $L_2$ is not regular.
    \end{itemize}
  \end{proof}
\end{problem}
%-------------------------------------------------------------------------------
% Problem 3.0
%-------------------------------------------------------------------------------
\begin{problem}{}{3}{0}
Given $0^*21^* \in L$ and $L$ is regular. By inspection we note the following pattern: 
 \begin{center}
    \begin{tabular}{| l | l | l | l | l | l | l |}
      \hline
      $L$                               & $L_{\frac{1}{3} - \frac{1}{3}}(L)$ \\ \hline
      0\underline{\textbf{2}}1          & 01                           \\ \hline
      00\underline{\textbf{21}}11       & 0011                         \\ \hline
      000\underline{\textbf{211}}111    & 000111                       \\ \hline
      0000\underline{\textbf{2111}}1111 & 00001111                     \\ \hline
      $\ \ \ \ \ \ \ \ \ \vdots$        & $\ \ \ \ \ \ \vdots$         \\ \hline
      $0^n21^{|xyz|-(n+1)}$                & $0^n1^n$                      \\ \hline
    \end{tabular} 
  \end{center}
Let $L_1 = \{ 0^n21^{|xyz|-(n+1)} \} \in L$. Regular languages are closed under 
intersection so then \newline $L_{\frac{1}{3} - \frac{1}{3}}(L_1) \cap L = \{ 0^n1^n \}$, but 
we know that \{$0^n1^n$\} not regular.
\end{problem}
\end{document}



 
% \begin{align}
% \end{align}
% \frac{}{}

