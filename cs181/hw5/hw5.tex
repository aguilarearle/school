\documentclass[11pt, letterpaper, onecolumn]{article}

% Imports
\usepackage[english]{babel}
\usepackage{fancyhdr}
\usepackage{extramarks}
\usepackage{amsmath, amsthm, amsfonts,mathtools, framed, wasysym}
\usepackage[shortcuts]{extdash} % Use \-/ for hyphenated words
\usepackage{environ}
\usepackage{fancyvrb}
\usepackage[top=1.00in, bottom=1.00in, left=0.75in, right=0.75in]{geometry}
\usepackage{algorithm}
\usepackage{algpseudocode}
\usepackage{accents}
% Pseudocode enabler
\usepackage{listings}
\usepackage{color}

\usepackage{breqn}

\usepackage[table]{xcolor}

\definecolor{dkgreen}{rgb}{0,0.6,0}
\definecolor{gray}{rgb}{0.5,0.5,0.5}
\definecolor{mauve}{rgb}{0.58,0,0.82}

\lstset{frame=tb,
  language=Java,
  aboveskip=3mm,
  belowskip=3mm,
  showstringspaces=false,
  columns=flexible,
  basicstyle={\small\ttfamily},
  numbers=none,
  numberstyle=\tiny\color{gray},
  keywordstyle=\color{blue},
  commentstyle=\color{dkgreen},
  stringstyle=\color{mauve},
  breaklines=true,
  breakatwhitespace=true,
  tabsize=3
}

% tikz
\usepackage{pgf}
\usepackage{tikz}
\usetikzlibrary{arrows,automata}

% Page
\headsep=0.3in
\linespread{1.15}
\setlength\parindent{0pt}

% Horizontal Rules
\newcommand{\hwRuleWidth}{0.4pt}
\renewcommand\headrulewidth{\hwRuleWidth}
\renewcommand\footrulewidth{\hwRuleWidth}

% Header and Footer
\pagestyle{fancy}
\lhead{\hwAuthor\ (\hwSection)}
\chead{\hwClass\ (\hwInstructor): \hwTitle}
\rhead{\firstxmark}
\lfoot{\lastxmark}
\cfoot{\thepage}
\newcommand{\enterProblemHeader}[2]{
  \nobreak\extramarks{}{Problem \arabic{#1}.\arabic{#2} continued on next page\ldots}\nobreak{}
  \nobreak\extramarks{Problem \arabic{#1}.\arabic{#2} (continued)}{Problem \arabic{#1}.\arabic{#2} continued on next page\ldots}\nobreak{}
}
\newcommand{\exitProblemHeader}[2]{
  \nobreak\extramarks{Problem \arabic{#1}.\arabic{#2} (continued)}{Problem \arabic{#1}.\arabic{#2} continued on next page\ldots}\nobreak{}
  \nobreak\extramarks{Problem \arabic{#1}.\arabic{#2}}{}\nobreak{}
}

% Counters
\setcounter{secnumdepth}{0}
\newcounter{partCounter}
\setcounter{partCounter}{1}
\newcounter{hwProblemCounter}
\newcounter{hwSubProblemCounter}
\numberwithin{equation}{hwProblemCounter}
\numberwithin{figure}{hwProblemCounter}

% Environments
\newenvironment{problem}[3] {
  \setcounter{hwProblemCounter}{#2}
  \setcounter{hwSubProblemCounter}{#3}
  \setcounter{partCounter}{1}
  \enterProblemHeader{hwProblemCounter}{hwSubProblemCounter}

  \large{\textbf{Problem #2.#3: #1}}\\
}{
  \exitProblemHeader{hwProblemCounter}{hwSubProblemCounter}
}

\newcommand{\question}[2]{\textbf{(#1)\ }\ #2\\}

\newenvironment{Proof}[1][Proof]
  {\proof[#1]\leftskip=1cm\rightskip=1cm}
  {\endproof}
%-------------------------------------------------------------------------------
% Assignmnet Variables
%-------------------------------------------------------------------------------
\newcommand{\hwTitle}{Homework 5}
\newcommand{\hwDueDate}{March 3, 2017}
\newcommand{\hwClass}{CS 181}
\newcommand{\hwInstructor}{Amit Sahai}
\newcommand{\hwAuthor}{}
\newcommand{\hwSection}{804501476}

% First Problem

\setcounter{hwProblemCounter}{1}
\setcounter{hwSubProblemCounter}{0}

%-------------------------------------------------------------------------------
\begin{document}
%-------------------------------------------------------------------------------
% TITLE
%-------------------------------------------------------------------------------
{\centering
  \Large{\textbf{\hwTitle}}\\
  \vspace{0.1in}\normalsize{\hwDueDate}\\
  \vspace{0.1in}\textbf{\hwAuthor} (\hwSection)\\
  \vspace{0.1in}\rule{\textwidth}{\hwRuleWidth}}\\


%-------------------------------------------------------------------------------
% Problem 1.0
%------------------------------------------------------------------------------
\begin{problem}{}{1}{0}
  We emulate a TM with FRTM as follows: \\
  \textbf{1) Seach Symbol:} We use two \blacksmiley{} to determine what the character
  on the left is. \\
  \textbf{2) The start state:} We write the input onto the tape and mark the
  first one with \blacksmiley{}.   
  \begin{center}
    \begin{tabular}{| l | l | l | l | l | l | l | l | l | l | l | l | l | l | l | l | l | l | l |}
      \hline
         \blacksmiley{} &       &       &      &         & \\ \hline
         $x_1$          & $x_2$ & $x_3$ & $x_4$ & $\dots$ & $x_n$  \\ \hline
    \end{tabular} 
  \end{center}
  \textbf{3) Move Right:} First remove \blacksmiley{} from current position then move
  right and write  \blacksmiley{} in new position. 
  \begin{center}
    \begin{tabular}{| l | l | l | l | l | l | l | l | l | l | l | l | l | l | l | l | l | l | l |}
      \hline
               & \blacksmiley{} &       &      &         & \\ \hline
         $x_1$ & $x_2$          & $x_3$ & $x_4$ & $\dots$ & $x_n$  \\ \hline
    \end{tabular} 
  \end{center}
  \textbf{4) Move Left:} We will determine the character on the left by guessing.
  We will move left and write \blacksmiley{}.
  \begin{center}
    \begin{tabular}{| l | l | l | l | l | l | l | l | l | l | l | l | l | l | l | l | l | l | l |}
      \hline
          \blacksmiley{}     & \blacksmiley{} &       &      &         & \\ \hline
         $x_1$ & $x_2$          & $x_3$ & $x_4$ & $\dots$ & $x_n$  \\ \hline
    \end{tabular} 
  \end{center}
\end{problem}
%-------------------------------------------------------------------------------
% Problem 2.0
%-------------------------------------------------------------------------------
\begin{problem}{}{2}{0}
  
  \textbf{Proof}  \\
    Suppose $Subset_{TM}$ is decidable $ \implies \exists \ D$ decides
    $Subset_{TM}$. We build a turing machine to decide
    $EQ_{TM} = \{(<M_1>,<M_2>) \ | \ M_1 \ and \ M_2 \ are \ TMs \ and \ L(M_1) = L(M_2)\}$ \\ \\
    
    \includegraphics[scale=0.65]{cs_181_hw5_p_2.png}

    Theorem 5.4 states that $EQ_{TM}$ is not decidable $\Rightarrow \Leftarrow$.
    Therefore $Subset_{TM}$ is not decidable.
\end{problem}

%-------------------------------------------------------------------------------
% Problem 3.0
%-------------------------------------------------------------------------------
\begin{problem}{}{3}{0}
  \textbf{(a)}
  We can include $L_1^{Diag}$ into the enumeration as such
  ($L_1^{Diag}, L_1, L_2, \dots$) and let $\alpha_i$ be a lexicographical
  enumeration of strings over $\{0,1\}$. We constructm
   \begin{align}
   L_2^{Diag} &= \{ \bar{x}_{ii} \ | \ \text{if} \ x_{ii} = 1\}
  \end{align}
  \begin{center}
    \begin{tabular}{| l | l | l | l | l | l | l | l | l | l | l | l | l | l | l | l | l | l | l |}
      \hline
                  &$\alpha_1$             &$\alpha_2$             &$\alpha_3$             &$\alpha_4$              &$\alpha_5$             &$\alpha_6$              &$\alpha_7$  &$\dots$ \\ \hline      
      $L_1^{diag}$ &\cellcolor{red}$x_{11}$ &$x_{12}$                &$x_{13}$                &$x_{14}$                &$x_{15}$                &$x_{16}$                &$x_{17}$     &$\dots$ \\ \hline
      $L_1$       &$x_{21}$                &\cellcolor{red}$x_{22}$ &$x_{23}$                &$x_{24}$                &$x_{25}$                &$x_{26}$                &$x_{27}$     &$\dots$ \\ \hline
      $L_2$       &$x_{31}$                &$x_{32}$                &\cellcolor{red}$x_{33}$ &$x_{34}$                &$x_{35}$                &$x_{36}$                &$x_{37}$     &$\dots$ \\ \hline
      $L_3$       &$x_{41}$                &$x_{42}$                &$x_{43}$                &\cellcolor{red}$x_{44}$ &$x_{45}$                &$x_{46}$                &$x_{47}$     &$\dots$ \\ \hline
      $L_4$       &$x_{51}$                &$x_{52}$                &$x_{53}$                &$x_{54}$                &\cellcolor{red}$x_{55}$ &$x_{56}$                &$x_{57}$     &$\dots$ \\ \hline
      $L_5$       &$x_{61}$                &$x_{62}$                &$x_{63}$                &$x_{64}$                &$x_{65}$                &\cellcolor{red}$x_{66}$ &$x_{67}$     &$\dots$ \\ \hline
      $L_6$       &$x_{71}$                &$x_{72}$                &$x_{73}$                &$x_{74}$                &$x_{75}$                &$x_{76}$                &\cellcolor{red}$x_{77}$    &$\dots$ \\ \hline
    \end{tabular} 
  \end{center}

  \textbf{Claim:} $L_2^{Diag}$ is not in the table.
  \begin{proof} $ $ \\
    Towards contradiction suppose $L_2^{Diag}$ is in the table. Then $\exists$
    $i$ s.t $L^{'} = x_{1i}x_{2i}x_{3i} \dots \in L_2^{Diag}$. Look at position $x_{ii}$.
    If $x_{ii}$ is in $L^{'}$ then by definition it is not in
    $L_2^{diag}$. Therfore $L_2^{Diag}$ is no in $L_i \ \forall \ i \ge i$ and
    $L_2^{diag} \ne L_1^{diag}$.
  \end{proof}
  \textbf{(b)} \\ 
  \textbf{Properties:} 
  \begin{itemize} 
  \item[1] $\forall i,j, \ i \ne j, \ L_i^{diag} \ne L_j^{diag}$
  \item[2] $\forall j \ \l_i^{diag} \ne L_j$  
  \end{itemize}
  Using the method in the previous problem we can construct new $L_i^{Diag}$'s
  and include them into the enumeration
  $E=$($L_1^{Diag},L_2^{Diag},L_3^{Diag},\dots,L_1, L_2,L_3, \dots$). We also note that
  by consruction of E,
  $\mathcal{L}^{diag} = \{L_1^{diag},L_2^{diag},L_3^{diag}, \dots \} \subseteq E$ \\
  \begin{center}
    \begin{tabular}{| l | l | l | l | l | l | l | l | l | l | l | l | l | l | l | l | l | l | l |}
      \hline
                  &$\alpha_1$             &$\alpha_2$             &$\alpha_3$             &$\alpha_4$              &$\alpha_5$             &$\alpha_6$              &$\alpha_7$  &$\dots$ \\ \hline      
      $L_1^{diag}$ &\cellcolor{red}$x_{11}$ &$x_{12}$                &$x_{13}$                &$x_{14}$                &$x_{15}$                &$x_{16}$                &$x_{17}$     &$\dots$ \\ \hline
      $L_2^{diag}$ &$x_{21}$                &\cellcolor{red}$x_{22}$ &$x_{23}$                &$x_{24}$                &$x_{25}$                &$x_{26}$                &$x_{27}$     &$\dots$ \\ \hline
      $L_3^{diag}$ &$x_{31}$                &$x_{32}$                &\cellcolor{red}$x_{33}$ &$x_{34}$                &$x_{35}$                &$x_{36}$                &$x_{37}$     &$\dots$ \\ \hline
      \dots \\ \hline 
      $L_1$       &$x_{41}$                &$x_{42}$                &$x_{43}$                &\cellcolor{red}$x_{44}$ &$x_{45}$                &$x_{46}$                &$x_{47}$     &$\dots$ \\ \hline
      $L_2$       &$x_{51}$                &$x_{52}$                &$x_{53}$                &$x_{54}$                &\cellcolor{red}$x_{55}$ &$x_{56}$                &$x_{57}$     &$\dots$ \\ \hline
      $L_3$       &$x_{61}$                &$x_{62}$                &$x_{63}$                &$x_{64}$                &$x_{65}$                &\cellcolor{red}$x_{66}$ &$x_{67}$     &$\dots$ \\ \hline
      $L_4$       &$x_{71}$                &$x_{72}$                &$x_{73}$                &$x_{74}$                &$x_{75}$                &$x_{76}$                &\cellcolor{red}$x_{77}$    &$\dots$ \\ \hline
    \end{tabular} 
  \end{center}
  \textbf{Claim: $\mathcal{L}^{diag}$ satisfies properties 1) and 2):} 
  \begin{proof} $ $ \\
    \textbf{Base case:} Let $E = (L_1, L_2, L_3, \dots)$ in class we showed that
    $L_1^{diag}$ is not in E and satisfies properties 1) and 2) so we add it
    $E = (L_1^{Diag},L_1, L_2,L_3, \dots)$. \\
    \textbf{Inductive Step:} Assume properties are true for n, s.t
    $E = (L_1^{Diag}, L_2^{Diag}, \dots , L_n^{diag},L_1, L_2,L_3, \dots)$ holds. Then
    we can construct $L_{n+1}^{diag}$, by definition the first
    $x_{11}x_{22} \dots x_{n-1n-1} = L_{n}^{diag}$. We add
    $x_{n+1} = \{ \bar{x}_{nn} \ | \ \text{if} \ x_{nn} = 1\}$. Now
    $L_{n+1}^{diag}$ holds for property 2) since $L_n^{diag}$ holds and the addition
    of $x_{n+1}$ makes it so that $L_{n+1}^{diag} \ne L_n^{diag}$ Therefore
    $L_{n+1}^{diag}$ is not equal to any language in the table.
  \end{proof}
  \textbf{(c)}
  We can arrange the set $E^{'}$ constructed in the previous problem as such
  $E^{'} = ( L_1, L_1^{diag}, L_2, L_2^{diag}, \dots)$, where we alternate
  between the languages $L_i$ and the languages $L_i^{diag}$. We construct
  $L^{superdiag} = \{\bar{x}_{ii} \ | \ \text{if} \ x_{ii} == 1\}$.
  \begin{center}
    \begin{tabular}{| l | l | l | l | l | l | l | l | l | l | l | l | l | l | l | l | l | l | l |}
      \hline
                  &$\alpha_1$             &$\alpha_2$             &$\alpha_3$              &$\alpha_4$             &$\alpha_5$      &$\alpha_6$              &$\alpha_7$  &$\dots$ \\ \hline
      $L_1$       &\cellcolor{red}$x_{11}$ &$x_{12}$                &$x_{13}$                &$x_{14}$                &$x_{15}$        &$x_{16}$              &$x_{17}$     &$\dots$ \\ \hline
      $L_1^{diag}$ &$x_{21}$                &\cellcolor{red}$x_{22}$ &$x_{23}$                &$x_{24}$                &$x_{25}$                &$x_{26}$                &$x_{27}$     &$\dots$ \\ \hline
      $L_2$       &$x_{31}$                &$x_{32}$                &\cellcolor{red}$x_{33}$ &$x_{34}$                &$x_{35}$ &$x_{36}$                &$x_{37}$     &$\dots$ \\ \hline
      $L_2^{diag}$ &$x_{41}$                &$x_{42}$                &$x_{43}$                &\cellcolor{red}$x_{44}$ &$x_{45}$                &$x_{46}$                &$x_{47}$     &$\dots$ \\ \hline
      \dots&\dots&\dots&\dots&\dots&\dots&\dots&\dots&\dots\\ \hline 
    \end{tabular} 
  \end{center}
  \textbf{Claim:} $L^{superdiag}$ is not in the set.
  \begin{proof}
    Towards contradiction suppose $L^{Superdiag} \in E$. Then $\exists$
    $i$ s.t $L^{'} = x_{1i}x_{2i}x_{3i} \dots \in L^{superdiag}$. Look at position
    $x_{ii}$. If $x_{ii}$ is in $L^{'}$ then by definition it is not in
    $L^{Superdiag} \ \Rightarrow \Leftarrow$.  Therfore $L^{Superdiag} \notin E$. 
  \end{proof}
  \textbf{(d)}
  We can append $L^{superdiag}$ to set $E^{'}$ of the previous problem and then
  create $L_2^{superdiag} = \{ \bar{x}_{ii} \ | \ \text{if} \ x_{ii} = 1\}$. \\
  \textbf{Claim:} $L_2^{superdiag}$ is not in the set.
  \begin{proof}
    Towards contradiction suppose $L_2^{Superdiag} \in E^{'}$. Then $\exists$
    $i$ s.t $L^{'} = x_{1i}x_{2i}x_{3i} \dots \in L_2^{superdiag}$. Look at position
    $x_{ii}$. If $x_{ii}$ is in $L^{'}$ then by definition it is not in
    $L^{Superdiag} \ \Rightarrow \Leftarrow$.  Therfore $L^{Superdiag} \notin E^{'}$. 
  \end{proof}
\end{problem}
\end{document}



 
% \begin{align}
% \end{align}
% \frac{}{}

