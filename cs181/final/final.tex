\documentclass[11pt, letterpaper, onecolumn]{article}

% Imports
\usepackage[english]{babel}
\usepackage{fancyhdr}
\usepackage{extramarks}
\usepackage{amsmath, amsthm, amsfonts,mathtools, framed, wasysym}
\usepackage[shortcuts]{extdash} % Use \-/ for hyphenated words
\usepackage{environ}
\usepackage{fancyvrb}
\usepackage[top=1.00in, bottom=1.00in, left=0.75in, right=0.75in]{geometry}
\usepackage{algorithm}
\usepackage{algpseudocode}
\usepackage{accents}
\usepackage{diagbox}


% Pseudocode enabler
\usepackage{listings}
\usepackage{color}

\usepackage{breqn}

\usepackage[table]{xcolor}

\definecolor{dkgreen}{rgb}{0,0.6,0}
\definecolor{gray}{rgb}{0.5,0.5,0.5}
\definecolor{mauve}{rgb}{0.58,0,0.82}

\lstset{frame=tb,
  language=Java,
  aboveskip=3mm,
  belowskip=3mm,
  showstringspaces=false,
  columns=flexible,
  basicstyle={\small\ttfamily},
  numbers=none,
  numberstyle=\tiny\color{gray},
  keywordstyle=\color{blue},
  commentstyle=\color{dkgreen},
  stringstyle=\color{mauve},
  breaklines=true,
  breakatwhitespace=true,
  tabsize=3
}

% tikz
\usepackage{pgf}
\usepackage{tikz}
\usetikzlibrary{arrows,automata}

% Page
\headsep=0.3in
\linespread{1.15}
\setlength\parindent{0pt}

% Horizontal Rules
\newcommand{\hwRuleWidth}{0.4pt}
\renewcommand\headrulewidth{\hwRuleWidth}
\renewcommand\footrulewidth{\hwRuleWidth}

% Header and Footer
\pagestyle{fancy}
\lhead{\hwAuthor\ (\hwSection)}
\chead{\hwClass\ (\hwInstructor): \hwTitle}
\rhead{\firstxmark}
\lfoot{\lastxmark}
\cfoot{\thepage}
\newcommand{\enterProblemHeader}[2]{
  \nobreak\extramarks{}{Problem \arabic{#1}.\arabic{#2} continued on next page\ldots}\nobreak{}
  \nobreak\extramarks{Problem \arabic{#1}.\arabic{#2} (continued)}{Problem \arabic{#1}.\arabic{#2} continued on next page\ldots}\nobreak{}
}
\newcommand{\exitProblemHeader}[2]{
  \nobreak\extramarks{Problem \arabic{#1}.\arabic{#2} (continued)}{Problem \arabic{#1}.\arabic{#2} continued on next page\ldots}\nobreak{}
  \nobreak\extramarks{Problem \arabic{#1}.\arabic{#2}}{}\nobreak{}
}

% Counters
\setcounter{secnumdepth}{0}
\newcounter{partCounter}
\setcounter{partCounter}{1}
\newcounter{hwProblemCounter}
\newcounter{hwSubProblemCounter}
\numberwithin{equation}{hwProblemCounter}
\numberwithin{figure}{hwProblemCounter}

% Environments
\newenvironment{problem}[3] {
  \setcounter{hwProblemCounter}{#2}
  \setcounter{hwSubProblemCounter}{#3}
  \setcounter{partCounter}{1}
  \enterProblemHeader{hwProblemCounter}{hwSubProblemCounter}

  \large{\textbf{Problem #2.#3: #1}}\\
}{
  \exitProblemHeader{hwProblemCounter}{hwSubProblemCounter}
}

\newcommand{\question}[2]{\textbf{(#1)\ }\ #2\\}

\newenvironment{Proof}[1][Proof]
  {\proof[#1]\leftskip=1cm\rightskip=1cm}
  {\endproof}
%-------------------------------------------------------------------------------
% Assignmnet Variables
%-------------------------------------------------------------------------------
\newcommand{\hwTitle}{Final}
\newcommand{\hwDueDate}{March 17, 2017}
\newcommand{\hwClass}{CS 181}
\newcommand{\hwInstructor}{Amit Sahai}
\newcommand{\hwAuthor}{}
\newcommand{\hwSection}{804501476}

% First Problem

\setcounter{hwProblemCounter}{1}
\setcounter{hwSubProblemCounter}{0}

%-------------------------------------------------------------------------------
\begin{document}
%-------------------------------------------------------------------------------
% TITLE
%-------------------------------------------------------------------------------
{\centering
  \Large{\textbf{\hwTitle}}\\
  \vspace{0.1in}\normalsize{\hwDueDate}\\
  \vspace{0.1in}\textbf{\hwAuthor} (\hwSection)\\
  \vspace{0.1in}\rule{\textwidth}{\hwRuleWidth}}\\


%-------------------------------------------------------------------------------
% Problem 1.0
%------------------------------------------------------------------------------
\begin{problem}{}{1}{0}
  Prove
  \begin{align}
    L_1 \diamond L_2 &= \{xy \ | \ x \in L_1, y \in L_2, \ \text{and} \ |x|=2|y| \}
  \end{align}
  is not context free. \\
  Let $L_1=\{0^{2n}1^{2n}\}$ and $L_2=\{0^n1^n\}$. Then,
  \begin{align}
     L_1 \diamond L_2 &= \{0^{2n}1^{2n}0^{n}1^{n} \ | \ x \in L_1, y \in L_2, \ \text{and} \ |x|=2|y| \}
  \end{align}
  \begin{proof} $ $ \\
    Towards contradiction assume $L_1 \diamond L_2$ is context-free.
    \begin{itemize}
      \setlength\itemsep{0em}
    \item[-] By the pumping lemma $\exists$ pumping length $p$.
    \item[-] Let $w=0^{2p}1^{2p}0^{p}1^{p} \in L_1 \diamond L_2$ and $|w| \ge p$.
    \item[-] By pumping lemma $0^{2p}1^{2p}0^{p}1^{p}=abcde$ s.t:
      \begin{itemize}
        \setlength\itemsep{0em}
        \item[1.] $|bd| \ \ge \ 1$ 
        \item[2.] $|bcd| \ \le \ p $
      \end{itemize}
    \item[Case 1:] $bcd=0^{\alpha}1^{\beta}$ (on the left side)
      \begin{itemize}
        \setlength\itemsep{0em}
        \item[$\bullet$] We pump down then we have either:  
          \begin{itemize}
            \item[1.] $ace=0^{2p-\alpha}1^{2p}0^{p}1^{p} \notin L_1 \diamond L_2$,
              since $2p-\alpha+2p =4p \implies \alpha=0$ and $1 \le \alpha \le p$
              $\Longrightarrow \Longleftarrow$
            \item[2.] $ace=0^{2p}1^{2p-\beta}0^{p}1^{p} \notin L_1 \diamond L_2$,
              since $2p-\beta+2p =4p \implies \beta=0$ and $1 \le \beta \le p$
              $\Longrightarrow \Longleftarrow$              
            \item[3.] $ace=0^{2p-\alpha}1^{2p-\beta}0^{p}1^{p} \notin L_1 \diamond L_2$ ,
              since $2p-\alpha + 2p - \beta = 4p \implies \alpha + \beta = 0$
              and $1 \le \alpha + \beta \le p$ $\Longrightarrow \Longleftarrow$              
          \end{itemize}          
      \end{itemize}
    \item[Case 2:] $bcd=0^{\alpha}1^{\beta}$ (on the right side)
        \begin{itemize}
        \setlength\itemsep{0em}
        \item[$\bullet$] We pump up then we have either:  
          \begin{itemize}
            \item[1.] $ace=0^{2p}1^{2p}0^{p+\alpha}1^{p} \notin L_1 \diamond L_2$,
              since $2(p+\alpha+p) = 4p \implies \alpha=0$ and $1 \le \alpha \le p$
              $\Longrightarrow \Longleftarrow$
            \item[2.] $ace=0^{2p}1^{2p}0^{p}1^{p+\beta} \notin L_1 \diamond L_2$,
              since $2(p+\beta+p) =4p \implies \beta=0$ and $1 \le \beta \le p$
              $\Longrightarrow \Longleftarrow$              
            \item[3.] $ace=0^{2p}1^{2p}0^{p+\alpha}1^{p+\beta} \notin L_1 \diamond L_2$ ,
              since $2(p+\alpha + p + \beta) = 4p \implies \alpha + \beta = 0$
              and $1 \le \alpha + \beta \le p$ $\Longrightarrow \Longleftarrow$              
          \end{itemize}          
        \end{itemize}
    \item[Case 3:] $bcd=1^{\alpha}0^{\beta}$ (middle)
        \begin{itemize}
        \setlength\itemsep{0em}
          \item[$\bullet$] We pump down then we have either:  
            \begin{itemize}
            \item[1.] $ace=0^{2p}1^{2p-\alpha}0^{p}1^{p} \notin L_1 \diamond L_2$,
              since $2p-\alpha+2p = 4p \implies \alpha=0$ and $1 \le \alpha \le p$
              $\Longrightarrow \Longleftarrow$
            \item[2.] $ace=0^{2p}1^{2p}0^{p-\beta}1^{p} \notin L_1 \diamond L_2$,
              since $2(p-\beta+p) =4p \implies \beta=0$ and $1 \le \beta \le p$
              $\Longrightarrow \Longleftarrow$
            \item[3.] $ace=0^{2p}1^{2p-\alpha}0^{p-\beta}1^{p} \notin L_1 \diamond L_2$,
              since $2p-\alpha+2p =2(p-\beta + p) \implies \beta=\alpha$. This
              is true if $\alpha = \beta = 0$ but $1 \le \beta+\alpha \le p$
              $\Longrightarrow \Longleftarrow$. We can also have that $\alpha = \beta$
              is true if $p$ is even and each is half of $p$. However this destroys
              symmetry in $L_1$ or $L_2$, $0^{2p}1^{2p-\alpha} \notin L_1$ or
              $0^{p-\beta}1^{p} \notin L_2$ $\Longrightarrow \Longleftarrow$.
          \end{itemize}          
        \end{itemize}        
    \end{itemize}
  \end{proof}
\end{problem}
\newpage
%-------------------------------------------------------------------------------
% Problem 2.0
%-------------------------------------------------------------------------------
\begin{problem}{}{2}{0}
  \section{(a)}
  Show that 
  \begin{align}
    HALT = \{(\langle M \rangle,x) \ | \ M \ \text{halts on input} \ x \}
  \end{align}
  is oracle decidable.
  \begin{proof} $ $ \\
    We construct OBTM $O(\langle M \rangle,x)$:
    \begin{itemize}
      \setlength\itemsep{0em}
      \item[-] $O$ writes $\langle M \rangle$ to machine tape and $w$ to input tape.
      \item[-] $O$ enters query state:
        \begin{itemize}
          \setlength\itemsep{0em}          
          \item[1:] $x \in L(M)$ then accept.
          \item[2:] $x \notin L(M)$ the reject.
        \end{itemize}
    \end{itemize}
    The query is immediate therefore if $x \notin L(M)$, we can reject without
    looping. Therefore $O$ always terminates thus it is a decider for HALT.
  \end{proof}
  \section{(b)}
  Show that 
  \begin{align}
    NEQ = \{(\langle M_1 \rangle,\langle M_2 \rangle) \ | L(M_1) \ne L(M_2) \}
  \end{align}
  is oracle recognizable.  
  \begin{proof} $ $ \\
    We construct OBTM $O( \langle M_1 \rangle,\langle M_2 \rangle)$: \\
    \textbf{Tapes:} \\
    In class we showed that a multiple tapes can be simulated with a single tape
    so we split the regular tape into 4 tapes $w_1$, $w_2$, $w_3$, and $w_4$.
    \begin{itemize}
      \setlength\itemsep{0em}
    \item[1] Write $\langle M_1 \rangle$ onto $w_1$
    \item[2] Write $\langle M_2 \rangle$ onto $w_2$
    \item[3] Will keep a binary count starting at 0 in $w_3$.
      \begin{itemize}
        \setlength\itemsep{0em}
        \item[-] We are assuming that all strings can be converted to binary.
      \end{itemize}
   \item[4] Will maintain a tuple starting at $(\$,\$)$ in $w_4$   
    \end{itemize}
    \textbf{States:}\\
    We will have states $S_1$, $S_{oracle}$, $S_3$, $S_4$, $q_{accept}$
    \begin{itemize}
    \item[$S_1$:]  Write contents of tape $w_1$ onto the machine tape and contents
      of $w_3$ onto the input tape. 
    \item[$S_{oracle}$:] Enter query state. After the result has been written onto
      the input tape write the result onto $w_4$ and move head of $w_4$ right. 
    \item[$S_3$:] Clear the machine tape and write the contents of $w_2$ onto
      machine tape. Clear input tape and write $w_3$ onto input tape. 
    \item[$S_4$:] Reset tape $w_4$ to (\$,\$) and increment value of tape $w_3$
      by one.   
    \end{itemize}
    \textbf{Transitions:}
    \begin{align}
      \delta(S_1, w_4=(\$,\$)) &\to (S_{oracle}) \\
      \delta(S_{oracle},  w_4=(x,\$)) &\to (S_3), \ x \in \{0,1\} \\
      \delta(S_3,  w_4=(x,\$)) &\to (S_{oracle}) \\
      \delta(S_3,  w_4=(x,y)) &\to (q_{accept}) \ \text{if} \ x\ne y \\
      \delta(S_3,  w_4=(x,y)) &\to (S_4) \ \text{if} \ x = y \\
      \delta(S_4,  w_4) &\to (S_1) 
    \end{align}
    This is still an OBTM as we have not changed the function of query
    and do not misuse the input and machine tapes. We make use of the regular
    tape like a tape of any TM and supplied states which allow us to recognize
    NEQ by determining whether a binary representations of a string is ever accepted
    by one an rejected by the other, if so we will accept. If not the machine will
    continue to increment counter and process the counter as strings and
    potentially loop if these two machines indeed accept the same language.
  \end{proof}
  \section{(c)}
  The language:
  \begin{align}
    Infinite = \{\langle M \rangle \ | \ |L(M)| = \infty \}
  \end{align}
  is not oracle recognizable. An OBTM that would try to recognize this language
  would have to check and infinite amount of strings to determine whether they all
  belong to $M$ and so it would never halt.
\end{problem}
\newpage
%-------------------------------------------------------------------------------
% Problem 3.0
%-------------------------------------------------------------------------------
\begin{problem}{}{3}{0}
  \section{(a)}
  Show that
  \begin{align}
    CLOSEBY = \{( \langle M_1 \rangle, \langle M_2 \rangle ) \ | \ \forall x \in L(M_1) \ \exists y \in L(M_2): ||x-y|| \le 1 \}
  \end{align}
  is undecidable.
  \begin{proof} 
    Assume for contradiction $\exists$ a decider D for CLOSEBY, create a TM N:
    \begin{itemize}
      \setlength\itemsep{0em}
    \item $N(w)$:
      \begin{itemize}
      \item[-] Let $u = \langle N \rangle$, by Recursion Theorem.
      \item[-] Let $\langle M \rangle$ be a TM that only accepts $\varepsilon$.
      \item[-] Run $D(u, \langle M \rangle)$:
        \begin{itemize}
        \item[1:] $D(u, \langle M \rangle)$: Accepts
          \begin{itemize}
          \item[$\cdot$] Accept all $w$
          \end{itemize}
        \item[2:] $D(u, \langle M \rangle)$: Rejects
          \begin{itemize}
          \item[$\cdot$] Accept $w$ iff $w=\varepsilon$
          \end{itemize}
        \end{itemize}
      \end{itemize}
    \end{itemize}
    
    \textbf{Analysis:}
    \begin{itemize}
      \setlength\itemsep{0em}
    \item[-] Case 1: $D(u, \langle M \rangle)$: \textbf{Accepts}
      $\implies L(N) = L(M) = \{\varepsilon \}$. This is true since the length of
      $\varepsilon$ is zero $\implies$ the only string in $L(N)$ is $\varepsilon$.
      However we accept all $w$
      $\implies L(N) = \{0,1\}^{*}$ and this is contradiction.
    \item[-] Case 2: $D(u, \langle M \rangle)$: \textbf{Rejects}
      $\implies L(N) \ne L(M)$, since $L(M)= \{\varepsilon \}$. However we
      only accept $\varepsilon \implies L(N) = \{\varepsilon \}$ and this is a contradiction.
    \end{itemize}
  \end{proof}
  \newpage
  \section{(b)}
  Show that
  \begin{align}
    CLOSEBY = \{( \langle M_1 \rangle, \langle M_2 \rangle ) \ | \ \forall x \in L(M_1) \ \exists y \in L(M_2): ||x-y|| \le 1 \}
  \end{align}
  is unrecognizable.
  \begin{proof}
  Assume for contradiction $\exists$ a recognizer R for CLOSEBY, create a TM N:
    \begin{itemize}
      \setlength\itemsep{0em}
    \item $N(w)$:
      \begin{itemize}
      \item[-] Let $u = \langle N \rangle$, by Recursion Theorem.
      \item[-] Let $\langle M \rangle$ be a TM that only accepts $\varepsilon$.
      \item[-] if $w = \varepsilon$ accept.
      \item[-] Run $R(u, \langle M \rangle)$:
        \begin{itemize}
        \item[1:] $R(u, \langle M \rangle)$: Accepts
          \begin{itemize}
          \item[$\cdot$] Accept all $w$
          \end{itemize}
        \item[2:] $R(u, \langle M \rangle)$: Rejects
          \begin{itemize}
          \item[$\cdot$] Accept $w$ iff $w=\varepsilon$
          \end{itemize}
        \end{itemize}
      \end{itemize}
    \end{itemize}

    \textbf{Analysis:}
    \begin{itemize}
      \setlength\itemsep{0em}
    \item[-] Case 1: $R(u, \langle M \rangle)$: \textbf{Accepts}
      $\implies L(N) = L(M) = \{\varepsilon \}$. This is true since the length of
      $\varepsilon$ is zero $\implies$ the only string in $L(N)$ is $\varepsilon$.
      However we accept all $w$
      $\implies L(N) = \{0,1\}^{*}$. $\Longrightarrow \Longleftarrow$
    \item[-] Case 2: $R(u, \langle M \rangle)$: \textbf{Rejects}
      $\implies L(N) \ne L(M)$, since $L(M)= \{\varepsilon \}$. However we
      only accept $\varepsilon \implies L(N) = \{\varepsilon \}$. $\Longrightarrow \Longleftarrow$
    \item[-] Case 3: $R(u, \langle M \rangle)$: \textbf{Loops} $\implies L(N) \ne L(M)$, since $L(M)= \{\varepsilon \}$.
      But by construction $L(N) = \{\varepsilon\}. \Longrightarrow \Longleftarrow$
    \end{itemize}  
  \end{proof}
  \newpage
  \section{(c)}
  Show that
  \begin{align}
    LEQ-HALT = \{( \langle M \rangle, \langle N \rangle )
    \ | \ \forall x \in \Sigma^{*}: M(x) \ \text{halts in fewer steps than} \ N(x) \}
  \end{align}
  is unrecognizable.
  \begin{proof}
    Assume for contradiction $LEQ-HALT$ is regular $\implies exists$ an
    enumerator E for $LEQ-HALT$.
    We construct M:
    \begin{itemize}
      \setlength\itemsep{0em}
    \item M(w):
      \begin{itemize}
      \item[-] Let $z = \langle M \rangle$, by Recursion Theorem.
      \item[-] $task_A =$ run $E(\varepsilon)$. 
      \item[-] $task_B =$ look at the sequence of produced by E.
        Wait until we find a tuple of form $(z,\langle N \rangle)$ is found.
      \item[-] run $task_A$ and $task_B$ in parallel.
        \begin{itemize}
          \item[$\cdot$] When $task_B$ finishes, run N(w).
        \end{itemize}
      \end{itemize}
    \end{itemize}
    The enumerator is producing a sequence $(\langle M \rangle,\langle N \rangle) \in LEQ-HALT$.
    However by construction $\langle M \rangle$ now takes longer to halt than $\langle N \rangle$.
    $\Longrightarrow \Longleftarrow$
  \end{proof}
\end{problem}

%-------------------------------------------------------------------------------
% Problem 4.0
%------------------------------------------------------------------------------
\newcommand{\mybs}{\backslashbox}
\begin{problem}{}{4}{0}
  Show that 
  \begin{align}
    ALICE = \{(M,R) \ | \ (M,R) \ \text{is recognizable}\}
  \end{align}
  is undecidable. \\
  
  We can convert the rules for Alice in Turing Land into tiling rules and we also convert a
  Turing machine to follow the rules of the tiling problem. \\

  \textbf{Tiles:} have the following elements within them
  \begin{itemize}
    \setlength\itemsep{0em}
  \item[1] $w_i \in \{ON,OFF\}$
  \item[2] $q_i \in Q$ if head is in cell or $\blacksmiley{}$ if head is not in cell.
  \item[3] $r$, the result of $i \ \% \ M$, where $\%$ is the modulo operator and i is the 
    position of the head on the tape.
  \item[4] Boundary tiles which are marked by an $X$.
  \end{itemize}
  \newpage
  
  \textbf{Rules:}

  Initial State: For simplicity assume $\textvisiblespace= \textvisiblespace \backslash \blacksmiley{}$\\

  \begin{center}
    \begin{tabular}{| l | l | l | l | l | l | l | l | l | l |}
      \hline      
      $X$&\mybs{$\textvisiblespace$}{$q_0$}&$\textvisiblespace$ \\ \hline
      $X$&$X$&$X$ \\ \hline
    \end{tabular}  
    \quad
    \quad
    \quad
    \begin{tabular}{| l | l | l | l | l | l | l | l | l | l |}
      \hline      
      \mybs{$\textvisiblespace$}{$q_0$}&$\textvisiblespace$&$\textvisiblespace$ \\ \hline
      $X$&$X$&$X$ \\ \hline
    \end{tabular}
    \quad
    \quad
    \quad 
    \begin{tabular}{| l | l | l | l | l | l | l | l | l | l |}
      \hline      
      $\textvisiblespace$&$\textvisiblespace$&$\textvisiblespace$ \\ \hline
      $X$&$X$&$X$ \\ \hline
    \end{tabular} 
  \end{center}

  The rules $(b_1, \dots, b_n) \to (c_1, \dots, c_n) $ in Alice in Turing Land correspond to tiling rules where different $n$ will create a different tiling size. These tiling rules correspond to a set of transitions for a TM. For simplicity assume that $w_i = w_i \backslash \blacksmiley,r$ we always start with the head in a position where $r=0$. \\
  
  Transition Right:
  \begin{center}
    \begin{tabular}{| l | l | l | l | l | l | l | l | l | l |}
      \hline
      $w_{i-1}$&$w_{i}$&$w_{i+1}^{'}$&$w_{i+2}^{'}$&$\cdots$&$w_{i+n}^{'}$   \\ \hline      
      $w_{i-1}$&$w_{i}$&$w_{i+1}^{'}$&$w_{i+2}^{'}$&$\cdots$&\mybs{$w_{i+n}$}{$q_{i+n}$}  \\ \hline
      $\cdots$&$\cdots$&$\cdots$&$\cdots$&$\cdots$&$\cdots$ \\ \hline
      $w_{i-1}$&$w_{i}$&$w_{i+1}^{'}$&$w_{i+2}^{'}$&$\cdots$&$w_{i+n}$  \\ \hline
      $w_{i-1}$&$w_{i}$&$w_{i+1}^{'}$&\mybs{$w_{i+2}$}{$q_{i+2}$}&$\cdots$&$w_{i+n}$\\ \hline      
      $w_{i-1}$&$w_{i}$&\mybs{$w_{i+1}$}{$q_{i+1}$}&$w_{i+2}$&$\cdots$&$w_{i+n}$\\ \hline
      $w_{i-1}$&\mybs{$w_{i}$}{0, $q_i$}&$w_{i+1}$&$w_{i+2}$&$\cdots$&$w_{i+n}$ \\ \hline
    \end{tabular}
  \end{center}
    \begin{align}
    \begin{cases}
      \delta(w_{i}, q_{i}) \to (w_i, q_{i+1}, R) \\
      \delta(w_{i+1}, q_{i+1}) \to (w_{i+1}^{'}, q_{i+2}, R) \\
      \delta(w_{i+2}, q_{i+2}) \to (w_{i+2}^{'}, q_{i+3}, R) \\
      \ \ \ \ \ \ \ \ \ \ \ \ \ \ \ \ \cdots \\
      \delta(w_{i+(n-1)}, q_{i+(n-1)}) \to (w_{i+(n-1)}^{'}, q_{i+n}, R) \\
      \delta(w_{i+n}, q_{i+n}) \to (w_{i+n}^{'}, q_{i+(n+1)}, R) 
    \end{cases}
  \end{align}

  \newpage
  Transition Left:

   \begin{center}
    \begin{tabular}{| l | l | l | l | l | l | l | l | l | l |}
      \hline
      $w_{i-n}^{'}$&$\cdots$&$w_{i-2}^{'}$&$w_{i-1}^{'}$&$w_{i}$&$w_{i+1}$   \\ \hline      
      \mybs{$w_{i-n}$}{$q_{i-n}$}&$\cdots$&$w_{i-2}^{'}$&$w_{i-1}^{'}$&$w_{i}$& $w_{i+1}$ \\ \hline
      $\cdots$&$\cdots$&$\cdots$&$\cdots$&$\cdots$&$\cdots$ \\ \hline
      $w_{i-n}$&$\cdots$&$w_{i-2}^{'}$&$w_{i-1}^{'}$&$w_i$&$w_{i+1}$  \\ \hline
      $w_{i-n}$&$\cdots$&\mybs{$w_{i-2}^{'}$}{$q_{i-2}$}&$w_{i-1}^{'}$&$w_i$&$w_{i+1}$\\ \hline      
      $w_{i-n}$&$\cdots$&$w_{i-2}$&\mybs{$w_{i-1}$}{$q_{i-1}$}&$w_{i}$&$w_{i+1}$\\ \hline
      $w_{i-n}$&$\cdots$&$w_{i-2}$&$w_{i-1}$&\mybs{$w_{i}$}{0, $q_{i}$}&$w_{i+1}$ \\ \hline
    \end{tabular} 
  \end{center}

  \begin{align}
    \begin{cases}
      \delta(w_{i}, q_{i}) \to (w_i, q_{i-1}, L) \\
      \delta(w_{i-1}, q_{i-1}) \to (w_{i-1}^{'}, q_{i-2}, L) \\
      \delta(w_{i-2}, q_{i-2}) \to (w_{i-2}^{'}, q_{i-3}, L) \\
      \ \ \ \ \ \ \ \ \ \ \ \ \ \ \ \ \cdots \\
      \delta(w_{i-(n-1)}, q_{i-(n-1)}) \to (w_{i-(n-1)}^{'}, q_{i-n}, L) \\
      \delta(w_{i-n}, q_{i-n}) \to (w_{i-n}^{'}, q_{i-(n+1)}, L) \\
    \end{cases}
  \end{align}


  Edge Cases: \\
  
  Head in left corner

  \begin{center}
    \begin{tabular}{| l | l | l | l | l | l | l | l | l | l |}
      \hline
      $w_{i}$&$w_{i+1}^{'}$&$w_{i+2}^{'}$&$\cdots$&$w_{i+n}^{'}$   \\ \hline      
      $w_{i}$&$w_{i+1}^{'}$&$w_{i+2}^{'}$&$\cdots$&\mybs{$w_{i+n}$}{$q_{i+n}$}  \\ \hline
      $\cdots$&$\cdots$&$\cdots$&$\cdots$&$\cdots$ \\ \hline
      $w_{i}$&$w_{i+1}^{'}$&$w_{i+2}^{'}$&$\cdots$&$w_{i+n}$  \\ \hline
      $w_{i}$&$w_{i+1}^{'}$&\mybs{$w_{i+2}$}{$q_{i+2}$}&$\cdots$&$w_{i+n}$\\ \hline      
      $w_{i}$&\mybs{$w_{i+1}$}{$q_{i+1}$}&$w_{i+2}$&$\cdots$&$w_{i+n}$\\ \hline
      \mybs{$w_{i}$}{0, $q_i$}&$w_{i+1}$&$w_{i+2}$&$\cdots$&$w_{i+n}$ \\ \hline
    \end{tabular} 
  \end{center}

  \begin{align}
    \begin{cases}
      \delta(w_{i}, q_{i}) \to (w_i, q_{i+1}, R) \\
      \delta(w_{i+1}, q_{i+1}) \to (w_{i+1}^{'}, q_{i+2}, R) \\
      \delta(w_{i+2}, q_{i+2}) \to (w_{i+2}^{'}, q_{i+3}, R) \\
      \ \ \ \ \ \ \ \ \ \ \ \ \ \ \ \ \cdots \\
      \delta(w_{i+(n-1)}, q_{i+(n-1)}) \to (w_{i+(n-1)}^{'}, q_{i+n}, R) \\
      \delta(w_{i+n}, q_{i+n}) \to (w_{i+n}^{'}, q_{i+(n+1)}, R) \\
    \end{cases}
  \end{align}
  \\ \\
  \begin{center}
    \begin{tabular}{| l | l | l | l | l | l | l | l | l | l |}
      \hline
      $w_{i}$&$w_{i+1}$&$w_{i+2}$&$\cdots$&$w_{i+n}$  \\ \hline
      $w_{i}$&$w_{i+1}$&$w_{i+2}$&$\cdots$&$w_{i+n}$\\ \hline      
      $w_{i}$&$w_{i+1}$&$w_{i+2}$&$\cdots$&$w_{i+n}$\\ \hline
      \mybs{$w_{i}$}{0, $q_i$}&$w_{i+1}$&$w_{i+2}$&$\cdots$&$w_{i+n}$ \\ \hline
    \end{tabular} 
  \end{center}

    \begin{align}
    \begin{cases}
      \delta(w_{i}, q_{i}) \to (w_i, q_{i-1}, L) \\
      \delta(w_{i-1}, q_{i-1}) \to (w_{i-1}^{'}, q_{i-2}, L) \\
      \delta(w_{i-2}, q_{i-2}) \to (w_{i-2}^{'}, q_{i-3}, L) \\
      \ \ \ \ \ \ \ \ \ \ \ \ \ \ \ \ \cdots \\
      \delta(w_{i-(n-1)}, q_{i-(n-1)}) \to (w_{i-(n-1)}^{'}, q_{i-n}, L) \\
      \delta(w_{i-n}, q_{i-n}) \to (w_{i-n}^{'}, q_{i-(n+1)}, L) \\
    \end{cases}
  \end{align}
  \\
  Head in right corner:

  \begin{center}
    \begin{tabular}{| l | l | l | l | l | l | l | l | l | l |}
      \hline
      $w_{i-n}^{'}$&$\cdots$&$w_{i-2}^{'}$&$w_{i-1}^{'}$&$w_{i}$ \\ \hline      
      \mybs{$w_{i-n}$}{$q_{i-n}$}&$\cdots$&$w_{i-2}^{'}$&$w_{i-1}^{'}$&$w_{i}$ \\ \hline
      $\cdots$&$\cdots$&$\cdots$&$\cdots$&$\cdots$ \\ \hline
      $w_{i-n}$&$\cdots$&$w_{i-2}^{'}$&$w_{i-1}^{'}$&$w_i$  \\ \hline
      $w_{i-n}$&$\cdots$&\mybs{$w_{i-2}^{'}$}{$q_{i-2}$}&$w_{i-1}^{'}$&$w_i$ \\ \hline      
      $w_{i-n}$&$\cdots$&$w_{i-2}$&\mybs{$w_{i-1}$}{$q_{i-1}$}&$w_{i}$ \\ \hline
      $w_{i-n}$&$\cdots$&$w_{i-2}$&$w_{i-1}$&\mybs{$w_{i}$}{0, $q_{i}$} \\ \hline
    \end{tabular} 
  \end{center}
  
    \begin{align}
    \begin{cases}
      \delta(w_{i}, q_{i}) \to (w_i, q_{i-1}, L) \\
      \delta(w_{i-1}, q_{i-1}) \to (w_{i-1}^{'}, q_{i-2}, L) \\
      \delta(w_{i-2}, q_{i-2}) \to (w_{i-2}^{'}, q_{i-3}, L) \\
      \ \ \ \ \ \ \ \ \ \ \ \ \ \ \ \ \cdots \\
      \delta(w_{i-(n-1)}, q_{i-(n-1)}) \to (w_{i-(n-1)}^{'}, q_{i-n}, L) \\
      \delta(w_{i-n}, q_{i-n}) \to (w_{i-n}^{'}, q_{i-(n+1)}, L) \\
    \end{cases}
  \end{align}
  \\ \\
  \begin{center}
    \begin{tabular}{| l | l | l | l | l | l | l | l | l | l |}
      \hline
      $w_{i-n}$&$\cdots$&$w_{i-2}$&$w_{i-1}$&$w_i$  \\ \hline
      $w_{i-n}$&$\cdots$&$w_{i-2}$&$w_{i-1}$&$w_i$ \\ \hline      
      $w_{i-n}$&$\cdots$&$w_{i-2}$&$w_{i-1}$&$w_{i}$ \\ \hline
      $w_{i-n}$&$\cdots$&$w_{i-2}$&$w_{i-1}$&\mybs{$w_{i}$}{0, $q_{i}$} \\ \hline
    \end{tabular} 
  \end{center}

  \begin{align}
    \begin{cases}
      \delta(w_{i}, q_{i}) \to (w_i, q_{i+1}, R) \\
      \delta(w_{i+1}, q_{i+1}) \to (w_{i+1}^{'}, q_{i+2}, R) \\
      \delta(w_{i+2}, q_{i+2}) \to (w_{i+2}^{'}, q_{i+3}, R) \\
      \ \ \ \ \ \ \ \ \ \ \ \ \ \ \ \ \cdots \\
      \delta(w_{i+(n-1)}, q_{i+(n-1)}) \to (w_{i+(n-1)}^{'}, q_{i+n}, R) \\
      \delta(w_{i+n}, q_{i+n}) \to (w_{i+n}^{'}, q_{i+(n+1)}, R) \\
    \end{cases}
  \end{align}
  
  Head occupying corner after transition:
  
  \begin{center}
    \begin{tabular}{| l | l | l | l | l | l | l | l | l | l |}
      \hline     
      \mybs{$w_{i+n}$}{r, $q_{i+n}$}&$w_{i+(n+1)}$&$w_{i+(n+2)}$&$\cdots$&$w_{i+(n+m)}$\\ \hline
      $w_{i+n}$&$w_{i+(n+1)}$&$w_{i+(n+2)}$&$\cdots$&$w_{i+(n+m)}$ \\ \hline
    \end{tabular}
  
    \begin{align}
      \begin{cases}      
        \delta(w_{i}, q_{i}) \to (w_i, q_{i+1}, R) \\
        \delta(w_{i+1}, q_{i+1}) \to (w_{i+1}^{'}, q_{i+2}, R) \\
        \delta(w_{i+2}, q_{i+2}) \to (w_{i+2}^{'}, q_{i+3}, R) \\
        \ \ \ \ \ \ \ \ \ \ \ \ \ \ \ \ \cdots \\
        \delta(w_{i+(n-1)}, q_{i+(n-1)}) \to (w_{i+(n-1)}^{'}, q_{i+n}, R) 
      \end{cases}
    \end{align}
    
    \begin{tabular}{| l | l | l | l | l | l | l | l | l | l |}
      \hline
      $w_{i-(n+m)}$&$\cdots$&$w_{i-(n+2)}$&$w_{i-(n+1)}$&\mybs{$w_{i-n}$}{r, $q_{i-n}$} \\ \hline
      $w_{i-(n+m)}$&$\cdots$&$w_{i-(n+2)}$&$w_{i-(n+1)}$&$w_{i-n}$ \\ \hline
    \end{tabular} 
  \end{center}

    \begin{align}
    \begin{cases}
      \delta(w_{i}, q_{i}) \to (w_i, q_{i-1}, L) \\
      \delta(w_{i-1}, q_{i-1}) \to (w_{i-1}^{'}, q_{i-2}, L) \\
      \delta(w_{i-2}, q_{i-2}) \to (w_{i-2}^{'}, q_{i-3}, L) \\
      \ \ \ \ \ \ \ \ \ \ \ \ \ \ \ \ \cdots \\
      \delta(w_{i-(n-1)}, q_{i-(n-1)}) \to (w_{i-(n-1)}^{'}, q_{i-n}, L) \\
    \end{cases}
  \end{align}

   In class we showed that if D decides the tiling then we can decide
   $M(\varepsilon)$ halts. Therefore since we can the rules of Alice in Turing
   Land into tiling rules and convert a Turing machine into a tiling instance
   then we cannot decide Alice in Turing land.
\end{problem}

\newpage
%-------------------------------------------------------------------------------
% Problem 5.0
%------------------------------------------------------------------------------
\begin{problem}{}{5}{0}
  First I will create a function
  \begin{align}
    Pattern(i,j,b) 
  \end{align}
  This function takes 3 parameters
  \begin{itemize}
    \setlength\itemsep{0em}
  \item[1] $i$ - the width of a block.
  \item[2] $j$ - the length of a block.
  \item[3] $b$ - the base we will use to produce the patterns.
  \end{itemize}
  It will produce patterns and map them to a block with dimensions $i \times j$.
  For example:
  \begin{align}
    Pattern(2,2,2) &= \bigg\{
    \begin{tabular}{|l|l|}
      \hline     
      0 & 0 \\ \hline
      0 & 0 \\ \hline
    \end{tabular} \ \
    \begin{tabular}{|l|l|}
      \hline     
      1 & 0 \\ \hline
      0 & 0 \\ \hline
    \end{tabular} \ \
    \begin{tabular}{|l|l|}
      \hline     
      0 & 1 \\ \hline
      0 & 0 \\ \hline
    \end{tabular} \ \
    \begin{tabular}{|l|l|}
      \hline     
      1 & 1 \\ \hline
      0 & 0 \\ \hline
    \end{tabular} \ \
    \cdots \ \
    \begin{tabular}{|l|l|}
      \hline     
      0 & 1 \\ \hline
      1 & 1 \\ \hline
    \end{tabular} \ \
    \begin{tabular}{|l|l|}
      \hline     
      1 & 1 \\ \hline
      1 & 1 \\ \hline
    \end{tabular}
    \bigg\}
  \end{align} 
  \section{(a)}
  We perform triangle scheduling on the length of the rug vs the pattern of
  the carpet.
  The graph is numbered with the order in which we build a rug with a specific
  pattern. This allows us to not traverse one dimension up to $\infty$. In addition
  since $i$ is countably infinite then $2^{1 \times i}$ is also countably
  infinite, therefore we also will not weave rugs for length $i$ for an infinite
  amount of time. Therefore we weave in one infinite day.
  \begin{lstlisting}[mathescape]
    Weaver 1:
    for i from 1 $\to \infty$: // Tile size $1 \times i$
        for p in $pattern(1,i,2)$: /* Here we traverse the pattern encoded with binary digits*/
            weave rug with pattern $p_i$
  \end{lstlisting}
  
  \begin{center}
    \begin{tabular}{|l|l|l|l|l|l|l|l|l|l|l|l|l|l|l|l|l|l|l|l|l|}
      \hline
      Length &&&&&&&&&&&&&&&&\\ \hline
      $\cdots$&$\cdots$&$\cdots$&$\cdots$&$\cdots$&$\cdots$&$\cdots$&$\cdots$&$\cdots$&$\cdots$&$\cdots$&$\cdots$&$\cdots$&$\cdots$&$\cdots$&$\cdots$&$\cdots$\\ \hline
      n=4&15&16&17&18&19&20&21&22&23&24&25&26&27&28&29&$\cdots$\\ \hline
      n=3&7&8&9&10&11&12&13&14&&&&&&&&$\cdots$\\ \hline
      n=2&3&4&5&6&&&&&&&&&&&&$\cdots$ \\ \hline
      n=1&1&2&&&&&&&&&&&&&&$\cdots$\\ \hline
      $P_{i}$&$P_1$&$P_2$&$P_3$&$P_4$&$P_5$&$P_6$&$P_7$&$P_8$&$P_9$&$P_{10}$&$P_{11}$&$P_{12}$&$P_{13}$&$P_{14}$&$P_{15}$&$\cdots$ \\ \hline
    \end{tabular} 
  \end{center}
  \newpage
  \section{(b)}
  We are extending the triangle scheduling to one more dimension $j$, the width
  of a rug. This triangle scheduling is on 3 dimensions. Creating a triangular
  plane and then we triangulate that plane. Once again this allows us to not
  spend all the time along one dimension and $j$ is also countably infinite so
  $2^{i\times j}$ is also countably infinite. Therefore we can weave in one infinite
  day.
  \begin{lstlisting}[mathescape]
    Weaver 2:
    for i from 1 $\to \infty$:
        for j from 1 $\to i$
            for p in  $pattern(i,j,2)$:
                weave rug for pattern $p_i$
  \end{lstlisting}
  
  \section{(c)}
  Here we take the example in part a) and now extend the base chosen in the
  $pattern$ function to be of the set $c_i$ of colors. Here once again the colors
  are countably infinite so the possible patterns for a specific length $i$
  is also countably infinite. Since we are performing triangle scheduling then
  we can weave in one infinite day.
  \begin{lstlisting}[mathescape]
    Weaver 3:
    for i from 1 $\to \infty$: 
        for c in 1 $\to i$: // Colors
            for p in $pattern(1,i,c)$: 
                weave rug for pattern $p_i$        
  \end{lstlisting}
  \section{(d)}
  We continue to extend the example to one more dimension. Here we add
  the weavers dimension and perform triangle scheduling. The number of weavers 
  is countably infinite therefore we will not traverse the weaver dimension 
  for an infinite amount of time. 
  \begin{lstlisting}[mathescape]
    
    Weaver 4:
    for i from 1 $\to \infty$: // i is the weaver $w_i$
        for j from 1 $\to i$: // j is the width
            for k from 1 $\to j$: // k is the length
                for c from 1 $\to k$: // Colors
                    for p in $pattern(j,k,c)$: 
                        weave rug for pattern $p_i$            
  \end{lstlisting}
\end{problem}

\newpage
%-------------------------------------------------------------------------------
% Problem Extra Credit
%------------------------------------------------------------------------------
\begin{problem}{Extra Credit}{6}{0}
  \section{a}
  
  We can apply cantors diagonalization argument on the width of the rugs and show
  that the weaver will always miss a rug of a specific width.
  
  \begin{center}
    \begin{tabular}{|l|l|l|l|l|l|l|l|l|l|l|l|l|l|l|l|l|l|l|l|l|}
      \hline
       &$pos_1$&$pos_2$&$pos_3$&$pos_4$&$pos_5$&$pos_6$&$pos_7$&$\cdots$ \\ \hline
      $w_1$&\cellcolor{red}$x_{11}$&$x_{12}$&$x_{13}$&$x_{14}$&$x_{15}$&$x_{16}$&$x_{17}$&$\cdots$ \\ \hline
      $w_2$&$x_{21}$&\cellcolor{red}$x_{22}$&$x_{23}$&$x_{24}$&$x_{25}$&$x_{26}$&$x_{27}$&$\cdots$  \\ \hline
      $w_3$&$x_{31}$&$x_{32}$&\cellcolor{red}$x_{33}$&$x_{34}$&$x_{35}$&$x_{36}$&$x_{37}$&$\cdots$  \\ \hline
      $w_4$&$x_{41}$&$x_{42}$&$x_{43}$&\cellcolor{red}$x_{44}$&$x_{45}$&$x_{46}$&$x_{47}$&$\cdots$  \\ \hline
      $w_5$&$x_{51}$&$x_{52}$&$x_{53}$&$x_{54}$&\cellcolor{red}$x_{55}$&$x_{56}$&$x_{57}$&$\cdots$  \\ \hline
      $w_6$&$x_{61}$&$x_{62}$&$x_{63}$&$x_{64}$&$x_{65}$&\cellcolor{red}$x_{66}$&$x_{67}$&$\cdots$  \\ \hline
      $\cdots$&$\cdots$&$\cdots$&$\cdots$&$\cdots$&$\cdots$&$\cdots$&$\cdots$&$\cdots$\\ \hline
    \end{tabular} 
  \end{center}
  Let
  
  \begin{align}
    w_{Diag} &= (x_{11}-1)(x_{22}-1)(x_{33}-1)(x_{44}-1)(x_{55}-1)  
  \end{align}
  \textbf{Claim:} $w_{Diag}$ is not in the table.
  \begin{proof}
    Towards contradiction suppose $w_{Diag}$ is in the table. Then $\exists \ i$ 
    s.t $w^{'}=x_{1i}x_{2i}x_{3i} \dots = w_{Diag}$. Look at position $x_{ii}$.
    If $x_{ii}$ is in $w^{'}$ then it is not in $w_{Diag}$. $\Longrightarrow \Longleftarrow$
  \end{proof}
  \section{b}
  We can map $\mathbb{R} \to \mathbb{R}^2$ as such
  \begin{align}
    Day \ &= 0.a_1a_2a_3a_4a_5a_6\dots \\
    Width \ &= 0.a_1a_3a_5\dots  \\
    Length \ &= 0.a_2a_4a_6 \dots \\
    Day &\to (Length, Width)
  \end{align}
  On a specific day we weave a rug with a length and width derived from the
  odd and even integers in the real representation of the day. 
  We calculate the set of patterns for a set of colors as follows
  \begin{align}
    c_k^{\lceil i \rceil \times \lceil j \rceil}
  \end{align}
  Then we perform the triangle scheduling.
  \begin{lstlisting}[mathescape]
    
    Weaver 4:
    for i,j in $day_k$: // j is the width
            for c from 1 $\to i$: // Colors
                for p in $pattern(i,j,c)$: 
                    weave rug for pattern $p_i$            
  \end{lstlisting}
   
  This will allow us to deliver the order in one infinite year.
  
\end{problem}



\end{document}



 
% \begin{align}
% \end{align}
% \frac{}{}

