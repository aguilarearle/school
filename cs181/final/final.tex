\documentclass[11pt, letterpaper, onecolumn]{article}

% Imports
\usepackage[english]{babel}
\usepackage{fancyhdr}
\usepackage{extramarks}
\usepackage{amsmath, amsthm, amsfonts,mathtools, framed, wasysym}
\usepackage[shortcuts]{extdash} % Use \-/ for hyphenated words
\usepackage{environ}
\usepackage{fancyvrb}
\usepackage[top=1.00in, bottom=1.00in, left=0.75in, right=0.75in]{geometry}
\usepackage{algorithm}
\usepackage{algpseudocode}
\usepackage{accents}
% Pseudocode enabler
\usepackage{listings}
\usepackage{color}

\usepackage{breqn}

\usepackage[table]{xcolor}

\definecolor{dkgreen}{rgb}{0,0.6,0}
\definecolor{gray}{rgb}{0.5,0.5,0.5}
\definecolor{mauve}{rgb}{0.58,0,0.82}

\lstset{frame=tb,
  language=Java,
  aboveskip=3mm,
  belowskip=3mm,
  showstringspaces=false,
  columns=flexible,
  basicstyle={\small\ttfamily},
  numbers=none,
  numberstyle=\tiny\color{gray},
  keywordstyle=\color{blue},
  commentstyle=\color{dkgreen},
  stringstyle=\color{mauve},
  breaklines=true,
  breakatwhitespace=true,
  tabsize=3
}

% tikz
\usepackage{pgf}
\usepackage{tikz}
\usetikzlibrary{arrows,automata}

% Page
\headsep=0.3in
\linespread{1.15}
\setlength\parindent{0pt}

% Horizontal Rules
\newcommand{\hwRuleWidth}{0.4pt}
\renewcommand\headrulewidth{\hwRuleWidth}
\renewcommand\footrulewidth{\hwRuleWidth}

% Header and Footer
\pagestyle{fancy}
\lhead{\hwAuthor\ (\hwSection)}
\chead{\hwClass\ (\hwInstructor): \hwTitle}
\rhead{\firstxmark}
\lfoot{\lastxmark}
\cfoot{\thepage}
\newcommand{\enterProblemHeader}[2]{
  \nobreak\extramarks{}{Problem \arabic{#1}.\arabic{#2} continued on next page\ldots}\nobreak{}
  \nobreak\extramarks{Problem \arabic{#1}.\arabic{#2} (continued)}{Problem \arabic{#1}.\arabic{#2} continued on next page\ldots}\nobreak{}
}
\newcommand{\exitProblemHeader}[2]{
  \nobreak\extramarks{Problem \arabic{#1}.\arabic{#2} (continued)}{Problem \arabic{#1}.\arabic{#2} continued on next page\ldots}\nobreak{}
  \nobreak\extramarks{Problem \arabic{#1}.\arabic{#2}}{}\nobreak{}
}

% Counters
\setcounter{secnumdepth}{0}
\newcounter{partCounter}
\setcounter{partCounter}{1}
\newcounter{hwProblemCounter}
\newcounter{hwSubProblemCounter}
\numberwithin{equation}{hwProblemCounter}
\numberwithin{figure}{hwProblemCounter}

% Environments
\newenvironment{problem}[3] {
  \setcounter{hwProblemCounter}{#2}
  \setcounter{hwSubProblemCounter}{#3}
  \setcounter{partCounter}{1}
  \enterProblemHeader{hwProblemCounter}{hwSubProblemCounter}

  \large{\textbf{Problem #2.#3: #1}}\\
}{
  \exitProblemHeader{hwProblemCounter}{hwSubProblemCounter}
}

\newcommand{\question}[2]{\textbf{(#1)\ }\ #2\\}

\newenvironment{Proof}[1][Proof]
  {\proof[#1]\leftskip=1cm\rightskip=1cm}
  {\endproof}
%-------------------------------------------------------------------------------
% Assignmnet Variables
%-------------------------------------------------------------------------------
\newcommand{\hwTitle}{Final}
\newcommand{\hwDueDate}{March 17, 2017}
\newcommand{\hwClass}{CS 181}
\newcommand{\hwInstructor}{Amit Sahai}
\newcommand{\hwAuthor}{}
\newcommand{\hwSection}{804501476}

% First Problem

\setcounter{hwProblemCounter}{1}
\setcounter{hwSubProblemCounter}{0}

%-------------------------------------------------------------------------------
\begin{document}
%-------------------------------------------------------------------------------
% TITLE
%-------------------------------------------------------------------------------
{\centering
  \Large{\textbf{\hwTitle}}\\
  \vspace{0.1in}\normalsize{\hwDueDate}\\
  \vspace{0.1in}\textbf{\hwAuthor} (\hwSection)\\
  \vspace{0.1in}\rule{\textwidth}{\hwRuleWidth}}\\


%-------------------------------------------------------------------------------
% Problem 1.0
%------------------------------------------------------------------------------
\begin{problem}{}{1}{0}
  Prove
  \begin{align}
    L_1 \diamond L_2 &= \{xy \ | \ x \in L_1, y \in L_2, \ \text{and} \ |x|=2|y| \}
  \end{align}
  is not context free. \\
  Let $L_1=\{0^{2n}1^{2n}\}$ and $L_2=\{0^n1^n\}$. Then,
  \begin{align}
     L_1 \diamond L_2 &= \{0^{2n}1^{2n}0^{n}1^{n} \ | \ x \in L_1, y \in L_2, \ \text{and} \ |x|=2|y| \}
  \end{align}
  \begin{proof} $ $ \\
    Towards contradiction assume $L_1 \diamond L_2$ is context-free.
    \begin{itemize}
      \setlength\itemsep{0em}
    \item[-] By the pumping lemma $\exists$ pumping length $p$.
    \item[-] Let $w=0^{2p}1^{2p}0^{p}1^{p} \in L_1 \diamond L_2$ and $|w| \ge p$.
    \item[-] By pumping lemma $0^{2p}1^{2p}0^{p}1^{p}=abcde$ s.t:
      \begin{itemize}
        \setlength\itemsep{0em}
        \item[1.] $|bd| \ \ge \ 1$ 
        \item[2.] $|bcd| \ \le \ p $
      \end{itemize}
    \item[Case 1:] $bcd=0^{\alpha}1^{\beta}$ (on the left side)
      \begin{itemize}
        \setlength\itemsep{0em}
        \item[$\bullet$] We pump down then we have either:  
          \begin{itemize}
            \item[1.] $ace=0^{2p-\alpha}1^{2p}0^{p}1^{p} \notin L_1 \diamond L_2$,
              since $2p-\alpha+2p =4p \implies \alpha=0$ and $1 \le \alpha \le p$
              $\Longrightarrow \Longleftarrow$
            \item[2.] $ace=0^{2p}1^{2p-\beta}0^{p}1^{p} \notin L_1 \diamond L_2$,
              since $2p-\beta+2p =4p \implies \beta=0$ and $1 \le \beta \le p$
              $\Longrightarrow \Longleftarrow$              
            \item[3.] $ace=0^{2p-\alpha}1^{2p-\beta}0^{p}1^{p} \notin L_1 \diamond L_2$ ,
              since $2p-\alpha + 2p - \beta = 4p \implies \alpha + \beta = 0$
              and $1 \le \alpha + \beta \le p$ $\Longrightarrow \Longleftarrow$              
          \end{itemize}          
      \end{itemize}
    \item[Case 2:] $bcd=0^{\alpha}1^{\beta}$ (on the right side)
        \begin{itemize}
        \setlength\itemsep{0em}
        \item[$\bullet$] We pump up then we have either:  
          \begin{itemize}
            \item[1.] $ace=0^{2p}1^{2p}0^{p+\alpha}1^{p} \notin L_1 \diamond L_2$,
              since $2(p+\alpha+p) = 4p \implies \alpha=0$ and $1 \le \alpha \le p$
              $\Longrightarrow \Longleftarrow$
            \item[2.] $ace=0^{2p}1^{2p}0^{p}1^{p+\beta} \notin L_1 \diamond L_2$,
              since $2(p+\beta+p) =4p \implies \beta=0$ and $1 \le \beta \le p$
              $\Longrightarrow \Longleftarrow$              
            \item[3.] $ace=0^{2p}1^{2p}0^{p+\alpha}1^{p+\beta} \notin L_1 \diamond L_2$ ,
              since $2(p+\alpha + p + \beta) = 4p \implies \alpha + \beta = 0$
              and $1 \le \alpha + \beta \le p$ $\Longrightarrow \Longleftarrow$              
          \end{itemize}          
        \end{itemize}
    \item[Case 3:] $bcd=1^{\alpha}0^{\beta}$ (middle)
        \begin{itemize}
        \setlength\itemsep{0em}
          \item[$\bullet$] We pump down then we have either:  
            \begin{itemize}
            \item[1.] $ace=0^{2p}1^{2p-\alpha}0^{p}1^{p} \notin L_1 \diamond L_2$,
              since $2p-\alpha+2p = 4p \implies \alpha=0$ and $1 \le \alpha \le p$
              $\Longrightarrow \Longleftarrow$
            \item[2.] $ace=0^{2p}1^{2p}0^{p-\beta}1^{p} \notin L_1 \diamond L_2$,
              since $2(p-\beta+p) =4p \implies \beta=0$ and $1 \le \beta \le p$
              $\Longrightarrow \Longleftarrow$
            \item[3.] $ace=0^{2p}1^{2p-\alpha}0^{p-\beta}1^{p} \notin L_1 \diamond L_2$,
              since $2p-\alpha+2p =2(p-\beta + p) \implies \beta=\alpha$. This
              is true if $\alpha = \beta = 0$ but $1 \le \beta \le p$
              $\Longrightarrow \Longleftarrow$. We can also have that $\alpha = \beta$
              is true if $p$ is even and each is half of $p$. However this destroys
              symmetry in $L_1$ and $L_2$, $0^{2p}1^{2p-\alpha} \notin L_1$ and
              $0^{p-\beta}1^{p} \notin L_2$ $\Longrightarrow \Longleftarrow$.
          \end{itemize}          
        \end{itemize}        
    \end{itemize}
  \end{proof}
\end{problem}
%-------------------------------------------------------------------------------
% Problem 2.0
%-------------------------------------------------------------------------------
\begin{problem}{}{2}{0}
  \section{(a)}
  Show that 
  \begin{align}
    HALT = \{(\langle M \rangle,x) \ | \ M \ \text{halts on input} \ x \}
  \end{align}
  is oracle decidable.
  \begin{proof} $ $ \\
    We construct OBTM $O(\langle M \rangle,x)$:
    \begin{itemize}
      \setlength\itemsep{0em}
      \item[-] $O$ writes $\langle M \rangle$ to machine tape and $w$ to input tape.
      \item[-] $O$ enters query state:
        \begin{itemize}
          \setlength\itemsep{0em}          
          \item[1:] $x \in L(M)$ then accept.
          \item[2:] $x \notin L(M)$ the reject.
        \end{itemize}
    \end{itemize}
    The query is immediate therefore if $x \notin L(M)$, we can reject without
    looping. Therefore $O$ always terminates thus it is a decider for HALT.
  \end{proof}
  \section{(b)}
  Show that 
  \begin{align}
    NEQ = \{(\langle M_1 \rangle,\langle M_2 \rangle) \ | L(M_1) \ne L(M_2) \}
  \end{align}
  is oracle recognizable.  
  \begin{proof} $ $ \\
    We construct OBTM $O( \langle M_1 \rangle,\langle M_2 \rangle)$: \\
    \textbf{Tapes:} \\
    In class we showed that a multiple tapes can be simulated with a single tape
    so we split the regular tape into 4 tapes $w_1$, $w_2$, $w_3$, and $w_4$.
    \begin{itemize}
      \setlength\itemsep{0em}
    \item[1] Write $\langle M_1 \rangle$ onto $w_1$
    \item[2] Write $\langle M_2 \rangle$ onto $w_2$
    \item[3] Will keep a binary count starting at 0 in $w_3$.
      \begin{itemize}
        \setlength\itemsep{0em}
        \item[-] We are assuming that all strings can be converted to binary.
      \end{itemize}
   \item[4] Will maintain a tuple starting at $(0,0)$ in $w_4$   
    \end{itemize}
    \textbf{States:}\\
    We will have states $S_1$, $S_{oracle}$, $S_3$, $S_4$, $q_{accept}$
    \begin{itemize}
    \item[$S_1$:]  Write contents of tape $w_1$ onto the machine tape and contents
      of $w_3$ onto the input tape. 
    \item[$S_{oracle}$:] Enter query state and write the contents of the first cell
      in the input tape onto $w_4$ and move head right. 
    \item[$S_3$:] Clear the machine tape and write the contents of $w_2$ onto
      machine tape. Clear input tape and write $w_3$ onto input tape. 
    \item[$S_4$:] Reset tape $w_4$ to (0,0) and increment tape $w_3$ by one.   
    \end{itemize}
    \textbf{Transitions:}
    \begin{align}
      \delta(S_1, w_4=(0,0)) &\to (S_{oracle}) \\
      \delta(S_{oracle},  w_4=(x,0)) &\to (S_3), \ x \in \{0,1\} \\
      \delta(S_3,  w_4=(x,y)) &\to (S_4) , \ \text{if} \ x=y \\
      \delta(S_3,  w_4=(x,y)) &\to (q_{accept}) \ \text{if} \ x\ne y \\
      \delta(S_4,  w_4) &\to (S_1) 
    \end{align}
    This is still an OBTM as we have not changed the function of query
    and do not misuse the input and machine tapes. We make use of the regular
    tape like a tape of any TM and supplied states which allow us to recogize
    NEQ by determining whether the binary representations of strings is ever accepted
    by one an rejected by the other if so we will accept. If not the machine will
    continue to process strins and potentially loop if these two machines
    indeed accept the same language.
  \end{proof}
  \section{(c)}
  The language:
  \begin{align}
    Infinite = \{\langle M \rangle \ | \ |L(M)| = \infty \}
  \end{align}
  is not oracle recognizable. An OBTM that would try to recognize this language
  would have to check and infinite amount of strings to determine whether they all
  belong to $M$ and so it would never halt.
\end{problem}
%-------------------------------------------------------------------------------
% Problem 3.0
%-------------------------------------------------------------------------------
\begin{problem}{}{3}{0}
  \section{(a)}
  Show that
  \begin{align}
    CLOSEBY = \{( \langle M_1 \rangle, \langle M_2 \rangle ) \ | \ \forall x \in L(M_1) \ \exists y \in L(M_2): ||x-y|| \le 1 \}
  \end{align}
  is undecidable.
  \begin{proof} 
    Assume for contradiction $\exists$ a decider D for CLOSEBY, create a TM N:
    \begin{itemize}
      \setlength\itemsep{0em}
    \item $N(w)$:
      \begin{itemize}
      \item[-] Let $u = \langle N \rangle$, by Recursion Theorem.
      \item[-] Let $\langle M \rangle$ be a TM that only accepts $\varepsilon$.
      \item[-] Run $D(u, \langle M \rangle)$:
        \begin{itemize}
        \item[1:] $D(u, \langle M \rangle)$: Accepts
          \begin{itemize}
          \item[$\cdot$] Accept all $w$
          \end{itemize}
        \item[2:] $D(u, \langle M \rangle)$: Rejects
          \begin{itemize}
          \item[$\cdot$] Accept $w$ iff $w=\varepsilon$
          \end{itemize}
        \end{itemize}
      \end{itemize}
    \end{itemize}
    
    \textbf{Analysis:}
    \begin{itemize}
      \setlength\itemsep{0em}
    \item[-] Case 1: $D(u, \langle M \rangle)$: \textbf{Accepts}
      $\implies L(N) = L(M) = \{\varepsilon \}$. This is true since the length of
      $\varepsilon$ is zero $\implies$ the only string in $L(N)$ is $\varepsilon$.
      However we accept all $w$
      $\implies L(N) = \{0,1\}^{*}$ and this is contradiction.
    \item[-] Case 2: $D(u, \langle M \rangle)$: \textbf{Rejects}
      $\implies L(N) \ne L(M)$, since $L(M)= \{\varepsilon \}$. However we
      only accept $\varepsilon \implies L(N) = \{\varepsilon \}$ and this is a contradiction.
    \end{itemize}
  \end{proof}
  \section{(b)}
  Show that
  \begin{align}
    CLOSEBY = \{( \langle M_1 \rangle, \langle M_2 \rangle ) \ | \ \forall x \in L(M_1) \ \exists y \in L(M_2): ||x-y|| \le 1 \}
  \end{align}
  is unrecognizable.
  \begin{proof}
  Assume for contradiction $\exists$ a recognizer R for CLOSEBY, create a TM N:
    \begin{itemize}
      \setlength\itemsep{0em}
    \item $N(w)$:
      \begin{itemize}
      \item[-] Let $u = \langle N \rangle$, by Recursion Theorem.
      \item[-] Let $\langle M \rangle$ be a TM that only accepts $\varepsilon$.
      \item[-] if $w = \varepsilon$ accept.
      \item[-] Run $R(u, \langle M \rangle)$:
        \begin{itemize}
        \item[1:] $R(u, \langle M \rangle)$: Accepts
          \begin{itemize}
          \item[$\cdot$] Accept all $w$
          \end{itemize}
        \item[2:] $R(u, \langle M \rangle)$: Rejects
          \begin{itemize}
          \item[$\cdot$] Accept $w$ iff $w=\varepsilon$
          \end{itemize}
        \end{itemize}
      \end{itemize}
    \end{itemize}

    \textbf{Analysis:}
    \begin{itemize}
      \setlength\itemsep{0em}
    \item[-] Case 1: $R(u, \langle M \rangle)$: \textbf{Accepts}
      $\implies L(N) = L(M) = \{\varepsilon \}$. This is true since the length of
      $\varepsilon$ is zero $\implies$ the only string in $L(N)$ is $\varepsilon$.
      However we accept all $w$
      $\implies L(N) = \{0,1\}^{*}$. $\Longrightarrow \Longleftarrow$
    \item[-] Case 2: $R(u, \langle M \rangle)$: \textbf{Rejects}
      $\implies L(N) \ne L(M)$, since $L(M)= \{\varepsilon \}$. However we
      only accept $\varepsilon \implies L(N) = \{\varepsilon \}$. $\Longrightarrow \Longleftarrow$
    \item[-] Case 3: $R(u, \langle M \rangle)$: \textbf{Loops} $\implies L(N) \ne L(M)$, since $L(M)= \{\varepsilon \}$.
      But by construction $L(N) = \{\varepsilon\}. \Longrightarrow \Longleftarrow$
    \end{itemize}  
  \end{proof}
  \newpage
  \section{(c)}
  Show that
  \begin{align}
    LEQ-HALT = \{( \langle M \rangle, \langle N \rangle )
    \ | \ \forall x \in \Sigma^{*}: M(x) \ \text{halts in fewer steps than} \ N(x) \}
  \end{align}
  is unrecognizable.
  \begin{proof}
    Assume for contradiction $LEQ-HALT$ is regular $\implies exists$ an
    enumarator E for $LEQ-HALT$.
    We construct M:
    \begin{itemize}
      \setlength\itemsep{0em}
    \item M(w):
      \begin{itemize}
      \item[-] Let $z = \langle x \rangle$, by Recursion Theorem.
      \item[-] $task_A =$ run $E(\varepsilon)$.
      \item[-] $task_B =$ look at the sequence of produced by E.
        Wait until we find a tuple of form $(z,\langle N \rangle)$ is found.
      \item[-] run $task_A$ and $task_B$ in parallel.
        \begin{itemize}
          \item[$\cdot$] When $task_B$ finishes, run N(w).
        \end{itemize}
      \end{itemize}
    \end{itemize}
    This is a contradicition because $\langle N \rangle$ now takes longer
    than $\langle N \rangle$ despite
    $(\langle M \rangle,\langle N \rangle) \in LEQ-HALT$.
  \end{proof}
\end{problem}

%-------------------------------------------------------------------------------
% Problem 4.0
%------------------------------------------------------------------------------
\begin{problem}{}{4}{0}
  Show that 
  \begin{align}
    ALICE = \{(M,R) \ | \ (M,R) \ \text{is recognizable}\}
  \end{align}
  is undecidable. \\
  We notice that this is an instance of the tiling problem.
  \begin{itemize}
  \item[1] \textbf{Not sure if this should be established:}
  \end{itemize}
  
  \textbf{Rules:}
  \begin{center}
    \begin{tabular}{| l | l | l | l | l | l | l | l | l | l | l | l | l | l | l | l | l | l | l |}
      \hline
      &  &  &  & & & & & \\ \hline      
      &  &  &  & & & & & \\ \hline
      &  &  &  & & & & & \\ \hline
      &  &  &  & & & & & \\ \hline     
    \end{tabular} 
  \end{center}

\end{problem}

\end{document}



 
% \begin{align}
% \end{align}
% \frac{}{}

