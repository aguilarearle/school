\documentclass[11pt, letterpaper, onecolumn]{article}

% Imports
\usepackage[english]{babel}
\usepackage{fancyhdr}
\usepackage{extramarks}
\usepackage{amsmath, amsthm, amsfonts,mathtools, framed, wasysym}
\usepackage[shortcuts]{extdash} % Use \-/ for hyphenated words
\usepackage{environ}
\usepackage{fancyvrb}
\usepackage[top=1.00in, bottom=1.00in, left=0.75in, right=0.75in]{geometry}
\usepackage{algorithm}
\usepackage{algpseudocode}
\usepackage{accents}
% Pseudocode enabler
\usepackage{listings}
\usepackage{color}

\usepackage{breqn}

\usepackage[table]{xcolor}

\definecolor{dkgreen}{rgb}{0,0.6,0}
\definecolor{gray}{rgb}{0.5,0.5,0.5}
\definecolor{mauve}{rgb}{0.58,0,0.82}

\lstset{frame=tb,
  language=Java,
  aboveskip=3mm,
  belowskip=3mm,
  showstringspaces=false,
  columns=flexible,
  basicstyle={\small\ttfamily},
  numbers=none,
  numberstyle=\tiny\color{gray},
  keywordstyle=\color{blue},
  commentstyle=\color{dkgreen},
  stringstyle=\color{mauve},
  breaklines=true,
  breakatwhitespace=true,
  tabsize=3
}

% tikz
\usepackage{pgf}
\usepackage{tikz}
\usetikzlibrary{arrows,automata}

% Page
\headsep=0.3in
\linespread{1.15}
\setlength\parindent{0pt}

% Horizontal Rules
\newcommand{\hwRuleWidth}{0.4pt}
\renewcommand\headrulewidth{\hwRuleWidth}
\renewcommand\footrulewidth{\hwRuleWidth}

% Header and Footer
\pagestyle{fancy}
\lhead{\hwAuthor\ (\hwSection)}
\chead{\hwClass\ (\hwInstructor): \hwTitle}
\rhead{\firstxmark}
\lfoot{\lastxmark}
\cfoot{\thepage}
\newcommand{\enterProblemHeader}[2]{
  \nobreak\extramarks{}{Problem \arabic{#1}.\arabic{#2} continued on next page\ldots}\nobreak{}
  \nobreak\extramarks{Problem \arabic{#1}.\arabic{#2} (continued)}{Problem \arabic{#1}.\arabic{#2} continued on next page\ldots}\nobreak{}
}
\newcommand{\exitProblemHeader}[2]{
  \nobreak\extramarks{Problem \arabic{#1}.\arabic{#2} (continued)}{Problem \arabic{#1}.\arabic{#2} continued on next page\ldots}\nobreak{}
  \nobreak\extramarks{Problem \arabic{#1}.\arabic{#2}}{}\nobreak{}
}

% Counters
\setcounter{secnumdepth}{0}
\newcounter{partCounter}
\setcounter{partCounter}{1}
\newcounter{hwProblemCounter}
\newcounter{hwSubProblemCounter}
\numberwithin{equation}{hwProblemCounter}
\numberwithin{figure}{hwProblemCounter}

% Environments
\newenvironment{problem}[3] {
  \setcounter{hwProblemCounter}{#2}
  \setcounter{hwSubProblemCounter}{#3}
  \setcounter{partCounter}{1}
  \enterProblemHeader{hwProblemCounter}{hwSubProblemCounter}

  \large{\textbf{Problem #2.#3: #1}}\\
}{
  \exitProblemHeader{hwProblemCounter}{hwSubProblemCounter}
}

\newcommand{\question}[2]{\textbf{(#1)\ }\ #2\\}

\newenvironment{Proof}[1][Proof]
  {\proof[#1]\leftskip=1cm\rightskip=1cm}
  {\endproof}
%-------------------------------------------------------------------------------
% Assignmnet Variables
%-------------------------------------------------------------------------------
\newcommand{\hwTitle}{Homework 6}
\newcommand{\hwDueDate}{March 3, 2017}
\newcommand{\hwClass}{CS 181}
\newcommand{\hwInstructor}{Amit Sahai}
\newcommand{\hwAuthor}{}
\newcommand{\hwSection}{804501476}

% First Problem

\setcounter{hwProblemCounter}{1}
\setcounter{hwSubProblemCounter}{0}

%-------------------------------------------------------------------------------
\begin{document}
%-------------------------------------------------------------------------------
% TITLE
%-------------------------------------------------------------------------------
{\centering
  \Large{\textbf{\hwTitle}}\\
  \vspace{0.1in}\normalsize{\hwDueDate}\\
  \vspace{0.1in}\textbf{\hwAuthor} (\hwSection)\\
  \vspace{0.1in}\rule{\textwidth}{\hwRuleWidth}}\\


%-------------------------------------------------------------------------------
% Problem 1.0
%------------------------------------------------------------------------------
\begin{problem}{}{1}{0}
  Prove
  \begin{align}
    COMP_{TM} &= \{ <M_1 , M_2> \ | \ L(M_1) = \overline{ L(M_2)}\}
  \end{align}
  
  \begin{proof}
    Towards contradiciton assume $COMP_{TM}$ is recognizable
    \begin{itemize}
      \setlength\itemsep{0em}
      \item $\exists$ turing machine $R$ which recognizes $COMP_{TM}$.
      \item Build turing machine $N$
      \item $N(y)$:
      \begin{itemize}
          \item Let $x = <N>$ // By Recursion Theorem
          \item Let $<P>$ be a turing machine that recognizes prime numbers.
          \item If $y = 4$ Accept.
          \item Run $R(x, <P>)$
            \begin{itemize}
                \item[1.] $R(x, <P>): Accepts$ \ \ //$R$ thinks $L(N)=L(\bar{P})$
                  \begin{itemize}
                    \item Run $P(y) \implies L(N) = L(P) \Longrightarrow \Longleftarrow$
                  \end{itemize}
                \item[2.] $R(x, <P>): Rejects$ \ \ //$R$ thinks $L(N) \ne L(\bar{P})$
                  \begin{itemize}
                    \item Run $\bar{P}(y) \implies L(N) = L(\bar{P}) \Longrightarrow \Longleftarrow$
                  \end{itemize}
                \item[3.] $R(x, <P>): Loops$  \ \ //$R$ thinks $L(N) \ne L(\bar{P})$
                  \begin{itemize}
                    \item But by construction $L(N) = {4} \in L(\bar{P}) \Longrightarrow \Longleftarrow$
                  \end{itemize}
            \end{itemize}
      \end{itemize}  
    \end{itemize}
  \end{proof}
\end{problem}
\newpage
%-------------------------------------------------------------------------------
% Problem 2.0
%-------------------------------------------------------------------------------
\begin{problem}{}{2}{0}
  Prove
  \begin{align}
    NEQ_{TM} &= \{<M_1, M_2> \ | \ L(M_1) \ne L(M_2) \}
  \end{align}

   \begin{proof}
    Towards contradiciton assume $NEQ_{TM}$ is recognizable
    \begin{itemize}
      \setlength\itemsep{0em}
      \item $\exists$ turing machine $R$ which recognizes $NEQ_{TM}$.
      \item Build turing machine $N$
      \item $N(y)$:
      \begin{itemize}
          \item Let $x = <N>$ // By Recursion Theorem
          \item Let $<M>$ be a turing machine that rejects everything.            
          \item If $y = \varepsilon$, accept.
          \item Run $R(x, <M>)$
            \begin{itemize}
                \item[1.] $R(x, <M>): Accepts$ \ \ //$R$ thinks $L(N) \ne L(M)$ 
                  \begin{itemize}
                    \item Reject all $y \implies L(N) = 0 \Longrightarrow \Longleftarrow$
                  \end{itemize}
                \item[2.] $R(x, <M>): Rejects$ \ \ //$R$ thinks $L(N) = L(M)$
                  \begin{itemize}
                    \item Accept all $y \implies L(N) \ne 0 \Longrightarrow \Longleftarrow$
                  \end{itemize}
                \item[3.] $R(x, <M>): Loops$  \ \ //$R$ thinks $L(N) = L(N)$
                  \begin{itemize}
                    \item But by construction $L(N) = \{ \varepsilon \} \implies L(N) \ne 0 \Longrightarrow \Longleftarrow$
                  \end{itemize}
            \end{itemize}
      \end{itemize}  
    \end{itemize}
  \end{proof}
\end{problem}

%-------------------------------------------------------------------------------
% Problem 3.0
%-------------------------------------------------------------------------------
\begin{problem}{}{3}{0}
  A certified language is a language over $\{0,1\}$ s.t there exists a turing
  machine M satisfying the following conditions:
  \begin{itemize}
  \setlength\itemsep{0em}    
  \item $\forall \ x \in L, \ \exists \ y \in \{0,1\}^{*} \ s.t \ M(x,y): accepts$
  \item $\forall \ x \notin L, \ and \ \forall \ y \in \{0,1\}^{*} \ s.t \ M(x,y): Rejects$  
  \end{itemize}
  \section{a: Show $Halt_{\varepsilon}$ is a certified language}
  Let $L^{'} = \{w \ | \ w \text{ is the number of steps in TM which halts on } \varepsilon \}$
  and let $y \in L^{'}$. \\
  We construct $M(x,y)$:
  \begin{itemize}
    \setlength\itemsep{0em}    
    \item Run $x(\varepsilon)$ one step at at time.
      \begin{itemize}
        \setlength\itemsep{0em}    
        \item If $x(\varepsilon)$ is in accept state after y steps: \textbf{accept}
        \item If the number of steps in $x(\varepsilon)$ is greater than y: \textbf{reject}
      \end{itemize}
  \end{itemize}
  The machine satisfies property 1) if $x \in L$ then $\exists \ y \in L^{'}$ which
  contains the number of steps to compute $x(\varepsilon)$. \\
  The machine also satisfies property 2) since if $x \notin L$ then
  there is no solution to $x(\varepsilon)$ since $x$ does not halt. Therefore
  $M(x,y)$ will reject for all $y$.
  \section{b)}
  The number of inputs to machine x is infinite and each one of these inputs
  produces a different number of steps for a computationl path. Therfore the
  number of steps is also infinite so we would never be able to verify if
  $x \in Halt_{all}$
  \section{c)}
\end{problem}
\end{document}



 
% \begin{align}
% \end{align}
% \frac{}{}

